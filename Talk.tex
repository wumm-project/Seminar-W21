\documentclass{beamer}
\usepackage{lsfolien}
\usepackage[english]{babel}
\usepackage[utf8]{inputenc}

\myfootline{Management and Cooperate Action -- A Short Summary}{Hans-Gert
  Gräbe}

\def\lt{\texttt{<}}
\def\gt{\texttt{>}}
\def\deg{$^\circ$C}

\newcommand{\ueberschrift}[1]{\begin{center}\bf #1\end{center}}

\title{Management and Cooperate Action\\[6pt]
  A Short Summary on our Seminar\\ at Leipzig University in S21
  \vskip1em}

\author{Prof. Dr. Hans-Gert Gräbe\\
\url{http://www.informatik.uni-leipzig.de/~graebe}}

\date{July 23, 2021}
\begin{document}

{\setbeamertemplate{footline}{}
\begin{frame}
  \titlepage
\end{frame}}

\begin{frame}{Seminar Setting}

  \ueberschrift{Seminar Objective}

  Better to understanding the term \emph{Management} on the background of the
  general topic \emph{"Contradictions and Management Methods"} of the WUMM
  Project.

\url{https://wumm-project.github.io/}
\end{frame}

\begin{frame}{Prior Knowledge of the Participants}
\begin{itemize}
\item Some experience in teaching management methods especially in the field
  of IT (ISO 9000, CMMI, Spice, TQM).
\item Study of systems theory approaches in socio-technical, socio-economic
  and socio-ecological contexts (W19)
\item Study of methods of analysis of contradictory requirement situations in
  the field of engineering (especially TRIZ; S19, W20)
\item Linking to aspects of a philosophy of technological development,
  especially in the tradition of Hegelian concepts of development and a
  materialist dialectic.
\item \textbf{No experience} neither with practical management nor with
  the world of management theories.
\end{itemize}
\end{frame}

\begin{frame}{How did we proceed?} 
\begin{itemize}
\item Joint research seminar with master students in computer science.
\item Selection of topics and literature based on recommendations from third
  parties and own investigation.
\item Each topic was prepared by a student for presentation, a handout was
  prepared and published beforehand, the presentation was intensively
  discussed in the seminar. This usually resultet in a first approach to the
  topic, as deeper theoretical knowledge cannot be expected from the students.
\item Already in the discussion this thematically narrow approach was usually
  broadened, overarching aspects were critically considered and lines of
  development were highlighted. 
\item The staff members played a particularly important role in this
  discussion due to their more detailed knowledge of the concepts developed so
  far. 
\end{itemize}
\end{frame}

\begin{frame}{How did we proceed?} 
\begin{itemize}
\item Individual aspects of this overarching reflection were summarised in
  \emph{Seminar Notes} (see the website of this workshop).
\item Now the student have to compile \emph{Seminar Papers} on this basis, in
  which the respective topic is to be deepened once again and placed in a
  larger context.
\item The seminar was accompanied by a \emph{Lecture} (for the slide stack,
  see the website of this workshop), in which the conceptual system developed
  so far, especially on 
  \begin{itemize}
  \item Technology, 
  \item Systems and Spaces of Action,
  \item Language -- Information -- Knowledge, 
  \item Digital transformation and Semantic Web, as well as
  \item Ccooperative Action,
  \end{itemize}
  were presented in more detail as a basis for further qualification of the
  seminar paper.
\end{itemize}
\end{frame}
\begin{frame}{Findings}

\textbf{1.}
There is a close connection between 
\begin{itemize}
\item the unfolding of an industrial mode of production,
\item technological development,
\item development of production-organisational instruments and
\item differentiation of professional profiles.
\end{itemize}
This concerns in particular the profession of an engineer and the profession
of a manager.

In both cases, educational structures (engineering schools, management
schools) emerged. Since the 1970s engineering schools have developed into an
essential component of a more general university education (technical
universities).

\end{frame}
\begin{frame}{Findings}

\textbf{2.}
This can be well mapped onto the \textbf{triad of our concept of technology},
consisting of 
\begin{itemize}
\item socially available processual knowledge,
\item institutionalised procedures,
\item private processural skills. 
\end{itemize}
In the studies on the concept of management, the focus was primarily on the
\textbf{process of institutionalisation} of production-organisational
processual knowledge.
\end{frame}
\begin{frame}{Findings}

\textbf{3.}
\textbf{Systemic} concepts play a central role in both the theory and practice
of \textbf{systematic} development of technical and social process landscapes.

They are a proven means to extract the \textbf{practically approved} from the
feedback cycle of "justified expectations -- experienced results" and to
institutionalise it in \textbf{approved practices}.
\end{frame}
\begin{frame}{Findings}

\textbf{4.}
Even though the beginnings date back to the second half of the 19th century,
these processes of differentiation essentially took place in the 20th century
(and are still taking place).

It did not develop in a linear fashion, but in stages with \textbf{clearly
  distinguishable paradigms}.
\end{frame}
\begin{frame}{Findings}

\textbf{5.}
The \textbf{first stage} is characterised by technological optimism,
mechanical-materialistic conceptions of processes and the attempt to transfer
the successes of machinic instrumentation to production-organisational
processes.  The factory should function like a machine.

This has a strong impact on management concepts of that time which were
essentially oriented towards preparing the workers for this algorithmically
driven organisation of production.

"One head and a thousand of hands".
\end{frame}
\begin{frame}{Findings}

\textbf{6.}  On the one hand, this impetus of controllability intensifies with
the availability of computer technologies, especially in the cybernetics of
the 1960s, but on the other hand it gets caught up in a complexity crisis.
This conflict essentially characterises the \textbf{second stage} of the
unfolding of management approaches.

In the latter, systemic concepts and dialectical-materialistic approaches to
process organisation are gaining importance, especially after the publication
of the "Limits of Growth" and further insights anchored in deeper scientific
understanding of the complexity of interrelations between the "natural" and
the "social".
\end{frame}
\begin{frame}{Findings}

\textbf{7.}
Both trends are intensifying in the current digital transformation, whereby
the front of the debate has shifted further to the sector of technology.

In engineering and computer science prevail mechanistic-algorithmic approaches
of a "Factory 4.0", of "Data Mining" ("Mining" the "Oil of the 21st century“)
or a "Human Brain Project".

They are confronted with the question of how the complexity of this processes
can be \textbf{practically mastered} and what instruments are available to
reduce complexity. Here again, systemic concepts \textbf{that are well
  anchored in domain knowledge} play a central role.
\end{frame}
\begin{frame}{Findings}

\textbf{8.}
\textbf{Sustainability} cannot be achieved without a solution of this
fundamental question of anchoring systemic and dialectical-materialist
("organismic") thinking in the foundations of the understanding of processes
of the industrial mode of production.

\emph{Scientific Thought as Planetary Phenomenon} (V.I.Vernadsky, 1938)

\emph{Learn to think in a new way} (Potsdam Manifesto, 2005)
\end{frame}
\begin{frame}{Findings}

\textbf{9.}  We encountered management in three forms: Management, Leadership
and Business Process Modelling.

We further encountered management on three levels of abstraction:

(1) \textbf{Management as the ability of the leader} to organise the area of
responsibility assigned to him or her in such a way that the required KPIs are
achieved.

For the foundation of this ability in a conceptual system, the feedback loop
is decisive between the manager's privately justified expectations and
privately experienced results.

Corresponding experiences "institutionalise" in the manager's world view and
form the basis of his or her actions.

\end{frame}
\begin{frame}{Findings}

(2) \textbf{Management as the ability of the company} to coordinate the
management processes (1) and to bundle them in a company-wide management
strategy.

For the foundation of this ability in a conceptual system, the feedback loop
is decisive between the justified expectations and the experienced results of
the group of managers in the company.  This is a cooperative process in the
company as action space.

Corresponding experiences institutionalise in company-wide specifications,
data collection processes and control structures and thus form the basis of
the company's actions.

\end{frame}
\begin{frame}{Findings}

(3) \textbf{General management theories} or theories of business process
modelling, in which the experiences of level (2) are generalised across
companies.

Level (2) is based on the systematisation of experiences at level (1), but on
the other hand provides the infrastructural prerequisites (e.g. through data
collection processes and controlling) under which certain forms of practical
activity at level (1) only become possible.

The same applies to the connection between levels (2) and (3).

\end{frame}
\begin{frame}{Findings}

\textbf{10.}  Precisely at these level boundaries run the processes of
institutional solidification of practically proven things in proven practices.

We observed a massive surplus of theory production at these two interfaces --
a lot of theory is produced, little of which becomes established.

\end{frame}
\end{document}
