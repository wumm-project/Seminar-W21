\documentclass{beamer}
\usepackage{lsfolien}
\usepackage[english]{babel}
\usepackage[utf8]{inputenc}

\myfootline{Sustainability, Environment, Management -- A Short
  Summary}{Hans-Gert Gräbe}

\def\lt{\texttt{<}}
\def\gt{\texttt{>}}
\def\deg{$^\circ$C}

\newcommand{\ueberschrift}[1]{\begin{center}\bf #1\end{center}}

\title{Sustainability, Environment, Management\\[6pt] A Short Summary on our
  Seminar\\ at Leipzig University in W21 \vskip1em}

\author{Prof. Dr. Hans-Gert Gräbe\\
\url{http://www.informatik.uni-leipzig.de/~graebe}}

\date{February 04, 2022}
\begin{document}

{\setbeamertemplate{footline}{}
\begin{frame}
  \titlepage
\end{frame}}

\begin{frame}{Seminar Objective}

  Learn more about 
  \begin{itemize}
  \item modern management appoaches in which \emph{common conceptualisations}
    and \emph{consensus-oriented decision-making processes} are central and of
    crucial importance for the success and ways of formation and consolidation
    of new systemic structures.
  \item connections between the dialectical resolution of contradictory
    requirement situations in the sense of TRIZ methodology and the emergence
    of common conceptual and notational worlds.
  \item the concepts of the \emph{Methodological School of Management} around
    G.P. Shchedrovitsky.
  \end{itemize}
\end{frame}

\begin{frame}{Prior Knowledge of the Participants}
\begin{itemize}
\item Some experience in teaching management methods especially in the field
  of IT (ISO 9000, CMMI, Spice, TQM).
\item Work on system theoretical approaches in socio-technical, socio-economic
  and socio-ecological contexts.
\item Study of methods of analysis of contradictory requirement situations in
  the field of engineering (especially TRIZ).
\item Some experience with philosophical argumentations, especially in the
  tradition of Hegelian concepts of development and materialist dialectics.
\item Partially work on Business Model patterns.
\end{itemize}
\end{frame}

\begin{frame}{How did we proceed?} 
\begin{itemize}
\item Joint research seminar with master students in computer science.
\item Selection of topics and literature based on recommendations from third
  parties and own investigation.
\item 5 topics were prepared by the students for presentation, a handout was
  compiled and published beforehand, the presentation was intensively
  discussed in the seminar. 8 topics were prepared and presented by staff
  members and also intensively discussed.
\item In the discussion the approach was usually broadened, overarching
  aspects were critically considered and lines of development were
  highlighted.
\item The staff members played a particularly important role in this
  discussion due to their more detailed knowledge of the concepts developed
  earlier.
\end{itemize}
\end{frame}

\begin{frame}{How did we proceed?} 
\begin{itemize}
\item Individual aspects of this overarching reflection were summarised in
  \emph{Seminar Notes} (see the website of this workshop).
\item Now the student have to compile \emph{Seminar Papers} on this basis, in
  which the respective topic is to be deepened once again and placed in a
  larger context.
\item The seminar was accompanied by a \emph{Lecture}, in which the conceptual
  system developed so far, especially on
  \begin{itemize}
  \item Technology, 
  \item Systems and Action Spaces,
  \item Language -- Information -- Knowledge, 
  \item Digital transformation and Semantic Web, as well as
  \item Ccooperative Action,
  \end{itemize}
  were presented in more detail.
\end{itemize}
\end{frame}
\begin{frame}{Findings}

\textbf{1.}
There is a close connection between 
\begin{itemize}
\item the unfolding of an industrial mode of production,
\item technological development,
\item development of production-organisational instruments and
\item differentiation of professional profiles.
\end{itemize}
This concerns in particular the profession of an engineer and the profession
of a manager.

In both cases, educational structures (engineering schools, management
schools) emerged. Since the 1970s engineering schools have developed into an
essential component of a more general university education (technical
universities).

\end{frame}
\begin{frame}{Findings}

\textbf{2.}
This can be well mapped onto the \textbf{triad of our concept of technology},
consisting of 
\begin{itemize}
\item socially available processual knowledge,
\item institutionalised procedures,
\item private processural skills. 
\end{itemize}
The focus of the seminar themes was primarily on the \textbf{process of
  institutionalisation} of production-organisational processual knowledge (ISO
9000, QM Frameworks, APQC Process Classification Framework, SCOR -- the Supply
Chain Operations Reference Model) and \textbf{proven practices} (BP
Landscaping, BM Patterns, BM Navigator).
\end{frame}
\begin{frame}{Findings}

\textbf{3.}  \textbf{Systemic} concepts play a central role in both the theory
and practice of \textbf{systematic} development of business process landscapes
and business models even if these are not always visible in the theoretical
models.

Systemic concepts are a proven means to extract the \textbf{practically
  approved} from the feedback cycle of "justified expectations -- experienced
results" and to institutionalise it in \textbf{approved practices}.
\end{frame}

\begin{frame}{Findings}

\textbf{4.}  We further encountered management on three levels of abstraction:

(1) \textbf{Management as the ability of the leader} to organise the area of
responsibility assigned to him or her in such a way that the required KPIs are
achieved.

For the foundation of this ability in a conceptual system, the feedback loop
is decisive between the manager's privately \emph{justified expectations} and
the privately \emph{experienced results} of the real business processes.

Corresponding experiences "institutionalise" in the manager's world view and
form the basis of his or her actions.

This is the level of operational management.

\end{frame}
\begin{frame}{Findings}

(2) \textbf{Management as the ability of the company} to coordinate the
management processes (1) and to bundle them in a company-wide management
strategy.

For the foundation of this ability in a conceptual system, the feedback loop
is decisive between the justified expectations and the experienced results of
the group of managers in the company.  This is a cooperative process in the
company as action space.

Corresponding experiences institutionalise in company-wide specifications,
data collection processes and control structures and thus form the basis of
the company's actions.

This is the level of strategic management.

\end{frame}
\begin{frame}{Findings}

(3) \textbf{General management theories} or theories of business process
modelling, in which the experiences of level (2) are generalised across
companies.

Level (2) is based on the systematisation of experiences at level (1), but on
the other hand provides the infrastructural prerequisites (e.g. through data
collection processes and controlling) under which certain forms of practical
activity at level (1) only become possible.

The same applies to the connection between levels (2) and (3).

ISO 9000, QM Frameworks, APQC Process Classification Framework, Business
Process Languages, Business Innovation along ISO 56000:2000 and Business Model
Patterns are located at this level.

\end{frame}
\begin{frame}{Findings}

\textbf{5.}  Precisely at these level boundaries run the processes of
institutional solidification of practically proven things in proven practices.

We observed a massive surplus of theory production at these two interfaces --
a lot of theory is produced, little of which becomes established.

\end{frame}
\begin{frame}{Findings}
\textbf{6.}  As a further level, we identified the theoretical elaboration of
practical \textbf{inter-company relations}.

This form of cooperative action in distributed multi-stakeholder structures is
characterised by a specific combination of cooperation and competition that
can hardly be grasped with a market-radical terminology only.

The \textbf{systematic} shaping of such relationships can well be
conceptualised as \textbf{systemic development} in the sense of
Shchedrovitsky's schematisation approach.

\end{frame}
\begin{frame}{Findings}
Of particular interest here were the approaches 
\begin{itemize}
\item SCOR -- the Supply Chain Operations Reference Model
\item Business Networks and Business Networking of the IMP Group (Ford, Mouzas
  2013)
\end{itemize}
Both approaches emphasise the importance of understanding the
\textbf{material} (substantial) interdependencies in such business networks
and thus real material flows of \textbf{use values} in qualitatively and
quantitatively determined dimensions.

In contrast to conceptualisations solely based on value propositions, which
Business Model Patterns in particular often reduce to, this leads to
fundamentally different priorities in modelling.

\end{frame}
\begin{frame}{Findings}
This corresponds exactly to the systemic premise that the \textbf{structures
  in delimited systems} are shaped to a large extent by the
\textbf{throughput} of energy, material and information.

It is emphasised that \textbf{Business Networking} as a \emph{process of
  targeted stabilisation, strenghtening and institutionalisation of such
  interdependencies} is more significant for the formation of cooperative
contexts than value propositions in the individual nodes of this network.

Such processes can well be conceptualised as "coping of (individual) problems"
in "small worlds".

The IMP group also emphasises the importance of the concept of
\textbf{Service} oriented at prosumer concepts for similar processes in a
wider world of such a Business Network.

\end{frame}
\end{document}
