\documentclass{beamer}
\usepackage{lsfolien,enumitem}
\usepackage[english]{babel}

\myfootline{Sustainability, Environment, Management -- Winter Term
  2021}{Hans-Gert Gr\"abe}

\newcommand{\ueberschrift}[1]{\begin{center}\bf #1\end{center}}

\title{Sustainability, Environment, Management \vskip1em}

\subtitle{Research Seminar in the Module 10-202-2309\\ for Master Computer
  Science}

\author{Prof. Dr. Hans-Gert Gräbe\\
\url{http://www.informatik.uni-leipzig.de/~graebe}}

\date{October 2021}
\begin{document}

{\setbeamertemplate{footline}{}
\begin{frame}
  \titlepage
\end{frame}}

\begin{frame}{Omnipresence of the System Concept}

The concept of a \emph{system} plays a prominent role in computer science when
it comes to database systems, software systems, hardware systems, accounting
systems, access systems, etc.

In general, computer science is regarded by a majority as the \vskip1em
\begin{quote}
  "science of the \emph{systematic} representation, storage, processing and
  transmission of information, especially their automatic processing using
  digital computers" (German Wikipedia).
\end{quote}
Also certain relevant professions such as the \emph{system architect} are in
high esteem by IT users.
\end{frame}

\begin{frame}{Omnipresence of the System Concept}
However, the significance of the concept of system extends far beyond the
field of computer science -- it is fundamental for all engineering sciences
and as \emph{Systems Engineering} with the ISO/IEC/IEEE-15288 standard
"Systems and Software Engineering", it is also the subject of international
standardisation processes.

Even more, the concept of systems also plays an important role in the
description of complex natural and cultural processes -- for instance in the
concept of an \emph{ecosystem}.
\end{frame}

\begin{frame}{Omnipresence of the System Concept}
While classical TRIZ focuses strongly on instrumentally feasible engineering
solutions, Systems Engineering\vskip1em 
\begin{quote}
  is an interdisciplinary field of engineering and engineering management that
  focuses on how to design, integrate, and manage complex systems over their
  life cycles. At its core, systems engineering utilizes systems thinking
  principles to organize this body of knowledge. The individual outcome of
  such efforts, an engineered system, can be defined as a combination of
  components that work in synergy to collectively perform a useful function.
  (English Wikipedia)
\end{quote}
\end{frame}

\begin{frame}{Sustainability, Environment, Management}

The issue of sustainability faces us with enormous challenges. One of the
clearest findings claims that there can be no more "business as usual". Our
industrial mode of production has to be influenced more clearly. Qualified
management of socio-technical, socio-economic and socio-ecological processes
is on the agenda.

In the past semester, we have been interested in various management theories
that are prevalent today.  We observed two types of theoretical approaches.
One kind of management theory (management by incentive, management by
objective, Taylor's principles of management, Minzberg on management) starts
from an individual manager with sufficient rights and appropriate authority
and reduces management to the question of which organisational, charismatic
and psychological means he or she can use to achieve externally set goals.

\end{frame}

\begin{frame}{Sustainability, Environment, Management}

The other kind of management theory (Toyota system, SMART approach, i* goal,
Business Process Definitions) focuses on various aspects of planning and
executing business processes. Russell Ackoff links systemic thinking and
management and thus embeds such modelling approaches of business processes
deeper into basic development processes of systems.

The practical implementation of a shift to a sustainable mode of production
requires both aspects -- charismatic leadership and planning built on solid
foundations. In our theory seminar, however, we will mainly focus on the
second point and \textbf{explore different foundations and approaches of a
  systemic-based business process modelling} in more detail.

\end{frame}

\begin{frame}{On the Notion of a System}

"Systemic-based" refers to a system concept that describes \textbf{the
  purpose-oriented interaction of viable components in a delimited context for
  the provision of emergent functions}. Such functions are not inherent to any
of the components, but result from their interaction within the context of a
defined throughput of energy, material and information, which is provided by
an -- in systemic terms -- "external world".

The connection between external throughput -- and thus access to available
resources -- and internal formation of structure is not only a technical
design principle in component-based architectures, but also frequently found
in nature. The examples range from the Bénard cell to complex metabolisms of
complicated living systems. It therefore makes a lot of sense to base
descriptions and modelling of real-world processes on such a system concept.

\end{frame}

\begin{frame}{Systems and Contradictions}\small

The WUMM project (WUMM stands in German for "Widersprüche und
Managementmethoden" -- contradictions and management methods) aims at a better
understanding of such systemic development processes.

Our starting point is TRIZ as a systematic innovation methodology derived from
engineering experience in contradictory requirement situations.

Today, similar demands for an experience-based \emph{systematic} approach are
also addressed in the field of management, which means that engineering
approaches and admissions are also there on the agenda.

With the field of "Business TRIZ", which has been unfolding for about 20
years, this transfer of experience is being actively promoted, embedded in
older management cultures and approaches.
\end{frame}

\begin{frame}{Systems and Contradictions}

In recent years, co-operative action by differently specialised experts has
become increasingly important.

In such interdisciplinary work contexts, the development of \emph{common
  conceptual systems} of sufficient performance proves to be a difficult
problem that can be supported by digital semantic technologies.

Parallel to these challenges \emph{agile approaches} play a major role in
recent years, not only in the field of management, but also increasingly in
the solution of socio-technical and engineering problems concerning ongoing
co-operative actions in multi-stakeholder contexts -- for example with the
concept of \emph{technical ecosystems}.
\end{frame}

\begin{frame}{The Seminar}

\textbf{In the seminar}, we want to learn more about modelling approaches of
business processes, especially with regard to contradictory requirement
situations that cannot be solved by compromises, but require a dialectical
resolution in the sense of the TRIZ methodology and the emergence of common
conceptual and notational worlds. 

It turns out that such modelling needs are particularly clear \textbf{when
  formalised applying suitable Semantic Web technologies}.

A special emphasis will be put on the work of the \emph{Methodological School
  of Management} and the Moscow Methodological Circle around
G.P. Shchedrovitsky.
\end{frame}

\begin{frame}{The Seminar}
The seminar is a \textbf{research seminar} in which we jointly explore
different aspects of co-operative action in Business Process Modelling
contexts.

With this seminar, we are approaching a topic that is new to us, which offers
the opportunity to participate in a joint academic explorative process on a
basis of equals.

This bears opportunities, but also challenges.  The students are expected to
actively participate in the seminar through seminar discussions, presentations
and last but not least by reading the relevant materials.

For the successful completion of the seminar, a topic has to be presented in
the seminar as discussion leader and a handout of 2--3 pages on the topic has
to be submitted in advance.
\end{frame}

\begin{frame}{The Seminar}

The seminar will be held weekly on Tuesdays 9-11 a.m. synchronously online.

Prior to each appointment participants have to study the assigned reading to
be in a position to discuss the problems in the seminar.

The seminar is moderated by a \emph{discussion leader}, who prepares a short
handout of 2--3 pages and makes it available to the participants in advance
\emph{before the seminar} (by Sunday evening).

Students of Leipzig University find more about the seminar in the Saxonian
e-learning platform OPAL -- Course W21.BIS.SIM.  The platform will be used for
organisational purposes only.

The \textbf{primary source for the seminar plan} is the (actual version of
the) file \texttt{Seminarplan.md} in the github repository \emph{Seminar-W21}.
\end{frame}

\begin{frame}{Course Structure}
The course includes
\begin{itemize}
\item[$\bullet$] A lecture "Modelling Sustainable Systems and Semantic Web"
\item[$\bullet$] A seminar "Sustainability, Environment, Management"
\item[$\bullet$] A TRIZ practical course.
\end{itemize}
\textbf{Note that the access to the e-learning system used in the TRIZ
  practical course is subject to a fee}. Details can be found in the forum of
the OPAL course.
\end{frame}

\begin{frame}{Course Structure}\vskip1em
In the \textbf{lecture} \emph{Modelling Sustainable Systems and Semantic Web}
(Thursdays 11-13 a.m.)  important concepts of our previous interdisciplinary
course programme such as\vspace{-1em}
\begin{itemize}
\item[$\bullet$] technology as a unity of socially available procedural
  knowledge, institutionalised procedures and private procedural skills,
\item[$\bullet$] sustainability requirements in systemic concepts,
\item[$\bullet$] digital changes and concepts of semantic web technologies,
\item[$\bullet$] concept and knowledge formation processes,
\item[$\bullet$] cooperative action, network economies and open culture
\end{itemize}\vspace{-1em}
will be developed in more detail.

The lecture and the seminar are not directly related to each other, but
conceptual frameworks developed in the lecture will be heavily present in the
seminar.\vskip3em
\end{frame}

\begin{frame}{Organisational Matters}
These course parts can be taken for credit in various combinations

\begin{itemize}
\item[1)] All three parts as In-depth Module 10-202-2309 (10 CP) "Modelling
  sustainable systems and semantic web".
  \begin{itemize}[noitemsep]
  \item[$\bullet$] \textbf{Prerequisites for examination:} Successfully
    completed seminar and practical course.
  \item[$\bullet$] \textbf{Examination:} Oral examination (30 min)
  \end{itemize}
\item[2)] Lecture and seminar as Seminar Module 10-202-2312(5 CP) "Applied
  Computer Science".
  \begin{itemize}[noitemsep]
  \item[$\bullet$] \textbf{Prerequisite for examination:} Successfully
    completed seminar.
  \item[$\bullet$] \textbf{Examination:} Seminar paper.
  \end{itemize}
\end{itemize}
\end{frame}

\begin{frame}{Organisational Matters}
\begin{itemize}
\item[3)] The practical course alone as Module 10-202-2012 (5 CP) "Current
  Trends in Computer Science".
  \begin{itemize}[noitemsep]
  \item[$\bullet$] \textbf{Prerequisite for examination:} Successfully
    completed practical course.
  \item[$\bullet$] \textbf{Examination:} Oral examination (30 min)
  \end{itemize}
\end{itemize}
More about this in OPAL \url{https://bildungsportal.sachsen.de/opal} in the
course W21.BIS.SIM.  There, please enrol first in the course and then in the
corresponding group.

You can access OPAL with the data of your studserv account.

\end{frame}

\begin{frame}{Organisational Matters}
You will find a more detailed lecture concept in the github repo
\url{https://github.com/wumm-project/Seminar-W21}.

\ueberschrift{Data protection}

We follow an Open Culture approach not only theoretically but also practically
and make course materials publicly available. This also applies to the course
materials you have to produce (presentations, seminar papers) as well as to
(annotated) chat sessions of the seminar discussions, in which your names are
also mentioned. \textbf{We assume your consent to this procedure if you do not
  explicitly object}. The discussions themselves are not recorded.

\end{frame}

\begin{frame}{Organisational Matters}

\begin{itemize}
\item[$\bullet$] Lecture: Thursdays 11:15-12:45, synchronous digital
\item[$\bullet$] Continuously updated lecture plan and list of references in
  the \texttt{Lecture/README.md} file in the github Repo.  
\item[$\bullet$] Further (mainly organisational) information also in the forum
  of the OPAL course.
\item[$\bullet$] Seminar: Tuesdays 9:15-10:45, synchronous digital
\item[$\bullet$] All events online in the BBB room BIS.SIM,
  \url{https://meet.uni-leipzig.de/b/gra-w2c-fhz-qnp}
\end{itemize}
\begin{center}\LARGE\bf
  Questions ?
\end{center}

See also \texttt{2021-10-12/README.md} for additional information about the
goal of the course. 

\end{frame}

\end{document}
