\documentclass{beamer}
\usepackage{lsfolien}
\usepackage[english]{babel}

\myfootline{Sustainability, Environment, Management -- Winter Term
  2021}{Hans-Gert Gr\"abe}

\newcommand{\ueberschrift}[1]{\begin{center}\bf #1\end{center}}

\title{Service Oriented Business Process Management \vskip1em}

\subtitle{Research Seminar in the Module 10-202-2309\\ for Master Computer
  Science}

\author{Prof. Dr. Hans-Gert Gräbe\\
\url{http://www.informatik.uni-leipzig.de/~graebe}}

\date{January 2021}
\begin{document}

{\setbeamertemplate{footline}{}
\begin{frame}
  \titlepage
\end{frame}}
\begin{frame}{The Topic of the Presentation} 
  
In the previous seminars, strategic corporate planning (business model
design) was considered from the perspective of an individual company and its
world conceptualisation.

However, these world conceptualisations are in turn interdependent and lead to
\emph{practical dependencies} between the execution of the individual business
models.

These dependencies can in turn be conceptualised, leading to \emph{systemic
  development processes at the inter-company level}.

This further dimension of cooperative action is addressed in (Ford, Mouzas
2013) taking the perspective of a rigorous developmental approach, as is also
the case with our concept of Cooperative Action.
\end{frame}

\begin{frame}{The Industrial Marketing and Purchasing (IMP) Group} 
  
From their Website \url{https://www.impgroup.org/about.php},
  
The IMP Group is an informal, international network of hundreds of scholars
who approach marketing, purchasing, innovation, technological development and
management from an interactive perspective, in a B2B and a B2C context. The
IMP Group's current work also includes research on public-private networks,
policy, and science-technology-business issues. ...

The IMP Group stands for three main features: 
\begin{itemize}
\item[(1)] a dynamic approach to economic exchange,
\item[(2)] empirically driven research on inter-organizational interactions,
  and
\item[(3)] an informal network of researchers forming a vibrant international
  community.
\end{itemize}\vskip1em
\end{frame}

\begin{frame}{The IMP Group}\small

Firstly, the IMP Group represents a dynamic approach to economic exchange,
which means that emphasis is placed on the interaction processes taking place
within and between business actors forming business relationships over time.
...

Secondly, the IMP Group represents a research tradition that places emphasis
on empirically-based studies of how companies actually do business and of the
various effects emerging when businesses and other organizations interact.
Based on the assumption of interdependent business actors, a hallmark of IMP
studies is that marketing, purchasing, technological development, innovation,
strategic management and logistics need to be investigated \emph{within the
  context of specific business relationships and networks}.

Thirdly, the IMP Group represents a large informal network of researchers. The
IMP Conference and the IMP Journal Seminar are important meeting places for
researchers from all over the world, all sharing an interactive perspective on
the business landscape. ...

\end{frame}

\begin{frame}{IMP Conceptualisation}

The following explains ideas from [Ford, Mouzas 2013].
  
\textbf{Business processes} are conceptualised as \emph{substantive}
interaction between activities, resources and the actors associated with them.

\textbf{Service Dominant Logic} (SDL) is largely \emph{conceptual} in
orientation.

The \textbf{heterogeneity}, the importance of \textbf{specific counterparts},
the \textbf{complexity} and \textbf{long-term nature} of business interaction
argue against generalisations about particular categories of actors such as
‘customers’, ‘suppliers’, ‘manufacturers’ or ‘retailers’ to conceptualise
their interactions.

IMP research is concerned to examine the idiosyncratic \textbf{Network
  Pictures} held by the actors within their \textbf{small world} of tight
functional dependencies which form the basis of their approaches to
interaction.\vskip1em
\end{frame}

\begin{frame}{IMP Conceptualisation}

Such analysis suggests that the small world of the business actors does not
exhibit the characteristics of a \emph{market} nor is it simply an
\emph{agglomeration of many markets}: Its structure is not one of independent
companies that have ease of entry or exit from the market or from their
dealings with specific counterparts as marketers or customers.

Instead, the analysis emphasises that many of the actors in this small world
are strongly interdependent with each other through their business.

The pattern of interdependencies across these \textbf{small worlds} and the
perspectives that arise from them form the \emph{context for continuing
  interaction} and the developments.

\end{frame}

\begin{frame}{IMP Conceptualisation}

This small world is a \emph{cooperative action space} as developed in the
lecture where "relational moments between actors shape the cooperative context
more than individual moments of individual actors“ with narrow, but permeable
boundaries.

This narrowness and permeability emphasises the limited horizons of all actors
and \emph{the importance of intermediaries} in interaction.

Interactions in business are not restricted to communication, negotiation or
to specific transactions but are \textbf{substantial}. In other words, they
involve a number of different aspects of the (practical) \emph{activities} and
(material) \emph{resources} of the actors which may be changed and transformed
and hence \emph{evolve} during action. 
\end{frame}

\begin{frame}{Small Worlds}
\emph{Example:} The development of ready-meals changes aspects of the
activities, resources and the actors involved in this small world. Some
activities such as the production systems of food producers becomes more or
less specialised towards the requirements of particular counterparts.
Resources, such as the stockholding facilities of producers, retailers and
logistics companies will have followed a particular \emph{path of investment}
or development and the actors themselves will have \emph{co-evolved}.

Co-evolution does not refer to an inevitable increase in the ‘closeness’ of
the relationships between interacting actors. Rather, it suggests that
\emph{the operations, characteristics and attitudes of business actors evolve
  as an outcome of their interactions} over time and
thus the set of relationships evolves itself.

Vargo and Lusch (2004, 2011): "Resources are not, they become”.

\end{frame}

\begin{frame}{Small Worlds in the Wider World}

All the actors are part of a \textbf{wider network} of substantial practical
dependencies.  However, each of these actors has a very restricted picture of
this ‘wider world’ and no direct interaction with most of the actors within
it.

For this reason, each actor will be dependent on \emph{service provision} by
some of its immediate counterparts who have relationships with or provide
access to others at a distance. Such service is similar to using
\emph{components off the shelf} (COTS) not in a technical but in a
production-organisational perspective.

For example, the producers of ready-meals are likely to depend on their
relationships with packaging companies to gain access to the activities and
resources of packaging materials companies. Similarly, logistics companies
will depend on their relationships with trailer suppliers to access the skills
of vehicle refrigeration contractors.

\end{frame}

\begin{frame}{Small Worlds in the Wider World}

This leads to a view of interaction in business relationships as a unique,
evolving, multifaceted process of \emph{‘problem-coping’ by and for all of the
  involved actors} (Webster 1965).

Shchedrovitsky: "If there are no problems, no management is required“.

The term ‘coping’ is used to emphasise the interactive and evolving nature of
business problems.

The complex, evolving and interactive nature of problem-coping also mean that
each actor has \emph{to conform to the status quo} (to "institutionalised
procedures" -- apply standard solutions to problems) in many aspects of its
relationships for which problems cannot immediately be addressed.

\end{frame}

\begin{frame}{Business Networking (Cooperation)}\small

Service-seeking and offering drives the process of \emph{activity
  specialisation}, \emph{division of labour by specialisation}, the \emph{path
  of resources} and the \emph{co-evolution of actors}.

The most significant problems that actors face concern \emph{the relationship
  structure in which they are embedded}. The business actor should be viewed
as a \textbf{node} within a network of relationships, so that what happens
\emph{outside} (i.e. inside the "small world") the actor and through its
relationships is likely to be more important in the evolution of that actor
than what happens inside.

IMP research uses the term \textbf{business networking} to refer to the
attempts of actors to change the structure and process of the relationships in
which they are involved.

It is through business networking that actors seek to cope with their problems
and those of others.

\end{frame}

\begin{frame}{Costs of Business Networking}\small

Short-term, \textbf{dyadic} problem coping may centre on a single transaction
involving the costs associated with transferring cash for one counterpart and
the benefits of service for the other (cost-benefit relation).

Short-term problem coping may involve working together to solve a particular
technical problem for \textbf{mutual benefit} (mutual benefit relation).

Short-term problem coping may appear to involve only one actor in benefits and
one in only costs. However, these \emph{short-term costs and benefits} received
will affect both actors \emph{long-term view} of their relationship.  The
long-term view considers short-term costs as \textbf{investment}.

In the longer term, problem coping will be based on \emph{investments} and
\emph{adaptations} by the counterparts (\textbf{synergy effects}) in one or
more aspects of the substance of their interaction.

\end{frame}

\begin{frame}{Costs of Business Networking}\small

Business actors commonly face issues over the \textbf{trade-offs between
  potential and actual short-term and long-term costs and benefits} of the
counterparts in relationships, expressed in terms of the \emph{extent and
  timing} of respective activity specialisation, resource path or actor
co-evolution.

It is likely that actors would perceive that much of the service actually
provided fell short of their expectations or exceeded them.

\textbf{Unforeseen contingencies} might explain this; late delivery of
services or products, not forthcoming cooperation, only partial adaptations,
payment less than expected. In contrast, technical assistance could produce
greater than anticipated cost savings or a cooperative development could
enhance an actor's relationship with a third party.

The existence of different perceptions among actors explains why profitable
business opportunities may exist whenever \textbf{prices fail to reflect the
  value}.
\end{frame}

\begin{frame}{Value}\small

The \textbf{value to a participant from service} is not a characteristic of
what is involved in it, whether product, services, payment or generalised
‘performance’.

An interactive and systemic conceptualisation of the business process requires
a refinement of this view of value
\begin{itemize}
\item \textbf{Value of problem coping:} As problem-coping process the value to
  each actor of a service is that \emph{actor's interpretation} of the worth
  of the service's contribution towards coping with one or more specific
  problems of the actor, \emph{identified by that actor}.

  Hence the value as the ‘perceived worth’ of the same service received by
  different respondents will be different and in all cases that value is time
  and problem-specific.

  Empirically, this nature of service value to a counterpart poses great
  difficulty for the provider in business interaction.
\end{itemize}
\end{frame}

\begin{frame}{Value}\small
\begin{itemize}

\item \textbf{Value and reciprocity:} The value of service is not determined
  solely by the receiving actor but also assessed by the service supplier.
  Each party makes its own assessment of the problem-specific value to
  themselves and to their counterparts of a service that they seek or provide.

  These multiple assessments form the basis for their approach to interaction
  in any single episode and to their expectations and intentions for future
  episodes and a relationship as a whole.

\item \textbf{Incidental value:} The business landscape is characterised by
  recurrent interactions between multiple actors in continuing relationships.

  Service provision and value creation in any of these may lead to incidental
  value to others, either positive or negative and in line or against the
  wishes of those involved.
\end{itemize}
\end{frame}

\begin{frame}{Service Value Levels}\small

The concept proposed by the authors connects
\begin{itemize}
\item \textbf{Business interaction} as problem-coping process of actions,
  reactions and re-reactions between actors,
\item \textbf{Services} as suuccessive outcomes of business interactions as
  perceived by the participants and
\item \textbf{Value} as actor's perception of the contribution of service to
  coping with a specific or general problem of particular actors.
\end{itemize}
  
Value of service may be identified at the following levels:
\begin{itemize}
\item \textbf{Episodic service value:} Service provision within a particular
  interaction episode. Value creation is the outcome of solving a particular
  problem rather than to conform to current ways of operating.
\end{itemize}
\end{frame}

\begin{frame}{Service Value Levels}\small

\begin{itemize}
\item \textbf{Relational service value:} Continuing or long-term service
  interaction in a dyadic relationship by developing the potential value of
  the relationship for future episodes.  Relational value at any one time
  depends on the interdependence of the counterpart's activities, the
  heterogeneity of their resources and the jointness of the actors.

\item \textbf{Service value in the small world:} Network effects for actors
  when to consolidate interactions within existing relationships, to change
  their pattern or to develop new relationships.

The costs and time involved in new relationship development often limit
networking opportunities to existing relationships.

However, problem coping in the business landscape \textbf{can never be wholly
  dyadic}.  The service offered by a single actor to another always depends on
service provision from other relationships.  An obvious example of this is
seen in the dependence of product suppliers on components supplied by others.
\end{itemize}
\end{frame}

\begin{frame}{Service Value Levels}\small

\begin{itemize}
\item \textbf{Service value in the wider world:} Because of an actor's lack of
  knowledge or established relationships in the wider world, this networking
  will either be based on \textbf{relationship development} or \textbf{service
    provision} by others.
\end{itemize}

\textbf{An example:} Extending the customer-base of detergent manufacturers by
reorienting offerings to provide service through coping with the environmental
concerns of consumers and the wish of retailers to be or to appear to be
environmentally friendly.

The small world: 
\begin{itemize}
\item Detergent manufacturers: Proctor \& Gamble, Henkel, Unilever
\item Ratailer: Tengelmann, Aldi, other retailers
\item Customers 
\end{itemize}
\end{frame}

\begin{frame}{Conclusions}

\textbf{1.} The conceptualisation of service in an interactive business
landscape allows to capture the inherent connectivity among interdependent
business actors.

This connectivity leads to a view of service as the successive and reciprocal
outcomes of recurrent interaction between multiple actors as perceived by the
involved business actors.

\textbf{2.} The idea of service in an interactive business landscape
transforms our view of the process of value creation and appropriation in
networks.

The value of a service is not confined to the provision by one company
(supplier) to an apparent receiver (customer).  Instead, service in the
business landscape is a \textbf{systemic process} producing different positive
and negative value for multiple actors, including those that appear only to be
providers.
\end{frame}

\begin{frame}{Conclusions}

The value of service is not confined to a single episode in which service
appears to have been provided.

A particular interaction episode that provides immediate value is also likely
to change the nature of the relationship in which it occurs, leading to
relationship value.

\textbf{3.} Thirdly, taking an interactive approach to service allows to
investigate the \textbf{dynamics} of problem coping and creation.

The evolution of problem-coping is observable through a continuing cycle of
recurring episodes and evaluations over time ("justified expectations" and
"experienced results" in the terminology of the lecture).
\end{frame}

\begin{frame}{Conclusions}

For the management, the idea of service in an interactive business landscape
emphasises the importance of \textbf{analysing} the evolving problems and
uncertainties of specific actors and the perceptions of those involved in the
interaction.

The concept of service highlights interaction in continuing relationships as
the successive, reciprocal, outcomes of action, reaction and re-reaction of
counterparts and thus the \textbf{evolution of the "small world"}.  This
requires perceptive analysis of relationship evolution, of the problems of the
company and its counterparts and a well developed, explicit but flexible
agenda.

Service in an interactive business landscape also involves a
\textbf{managerial reorientation} away from things, products and services and
towards the evolving problems of the company and its specific counterparts.
\end{frame}

\begin{frame}{Conclusions}

\textbf{Service provision} can range from obvious manifestations, such as the
payment of an invoice, the delivery of a product or the development of a new
technology to the subtle or complex, including the provision of advice or
reassurance, organisational transformation or intellectual assets, know-how
and expertise.

The nature of service delivery is \textbf{defined by the recipient} and its
value is determined by the problems of the recipient.

An understanding of the concept of service and value in an interactive
business landscape enables managers to \textbf{relate their own resources and
  activities} to those of others as the basis of coping with their respective
problems ("The whole is more than the sum of its parts").
  
\end{frame}
  
  
\end{document}
