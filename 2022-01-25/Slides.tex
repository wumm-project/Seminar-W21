\documentclass{beamer}
\usepackage{lsfolien}
\usepackage[english]{babel}

\myfootline{Sustainability, Environment, Management -- Winter Term
  2021}{Hans-Gert Gr\"abe}

\newcommand{\ueberschrift}[1]{\begin{center}\bf #1\end{center}}

\title{Service Oriented Business Process Management \vskip1em}

\subtitle{Research Seminar in the Module 10-202-2309\\ for Master Computer
  Science}

\author{Prof. Dr. Hans-Gert Gräbe\\
\url{http://www.informatik.uni-leipzig.de/~graebe}}

\date{January 2021}
\begin{document}

{\setbeamertemplate{footline}{}
\begin{frame}
  \titlepage
\end{frame}}

\begin{frame}{The Industrial Marketing and Purchasing (IMP) Group} 
  
From their Website \url{https://www.impgroup.org/about.php},
  
The IMP Group is an informal, international network of hundreds of scholars
who approach marketing, purchasing, innovation, technological development and
management from an interactive perspective, in a B2B and a B2C context. The
IMP Group's current work also includes research on public-private networks,
policy, and science-technology-business issues. ...

The IMP Group stands for three main features: 
\begin{itemize}
\item[(1)] a dynamic approach to economic exchange,
\item[(2)] empirically driven research on inter-organizational interactions,
  and
\item[(3)] an informal network of researchers forming a vibrant international
  community.
\end{itemize}\vskip1em
\end{frame}

\begin{frame}{The IMP Group}\small

Firstly, the IMP Group represents a dynamic approach to economic exchange,
which means that emphasis is placed on the interaction processes taking place
within and between business actors forming business relationships over time.
...

Secondly, the IMP Group represents a research tradition that places emphasis
on empirically-based studies of how companies actually do business and of the
various effects emerging when businesses and other organizations interact.
Based on the assumption of interdependent business actors, a hallmark of IMP
studies is that marketing, purchasing, technological development, innovation,
strategic management and logistics need to be investigated \emph{within the
  context of specific business relationships and networks}.

Thirdly, the IMP Group represents a large informal network of researchers. The
IMP Conference and the IMP Journal Seminar are important meeting places for
researchers from all over the world, all sharing an interactive perspective on
the business landscape. ...

\end{frame}

\begin{frame}{IMP Conceptualisation}

The following explains ideas from [Ford, Mouzas 2013].
  
\textbf{Business processes} are conceptualised as \emph{substantive}
interaction between activities, resources and the actors associated with them.

\textbf{Service Dominant Logic} (SDL) is largely \emph{conceptual} in
orientation.

The emphasis on heterogeneity, the importance of specific counterparts, the
complexity and long-term nature of business interaction militate against
generalisations about particular categories of actors such as ‘customers’,
‘suppliers’, ‘manufacturers’ or ‘retailers’, about their interactions.

IMP research is concerned to examine the idiosyncratic \emph{Network Pictures}
held by the actors within their 'small world' which form the basis of their
approaches to interaction.
\end{frame}

\begin{frame}{IMP Conceptualisation}

Such analysis suggests that the small world of the business actors does not
exhibit the characteristics of a \emph{market} nor is it simply an
\emph{agglomeration of many markets}: Its structure is not one of independent
companies that have ease of entry or exit from the market or from their
dealings with specific counterparts as marketers or customers.

Instead, the analysis emphasises that many of the actors in this small world
would have become interdependent with each other through their business
together.

The pattern of interdependencies across these small worlds and the
perspectives that arise from them form the \emph{context for continuing
  interaction} and the developments.

\end{frame}

\begin{frame}{IMP Conceptualisation}

This analysis also emphasises the narrow, but permeable boundaries of
each actor's small world.

This narrowness and permeability emphasises the limited horizons of all actors
and \emph{the importance of intermediaries} in interaction.

Interactions in business are not restricted to communication, negotiation or
to specific transactions but are \emph{substantial}. In other words, they
involve a number of different aspects of the (practical) \emph{activities} and
(material) \emph{resources} of the actors which may be changed and transformed
and hence \emph{evolve} during action. 
\end{frame}

\begin{frame}{Small Worlds}
\emph{Example:} The development of ready-meals changes aspects of the
activities, resources and the actors involved in this small world. Some
activities such as the production systems of food producers becomes more or
less specialised towards the requirements of particular counterparts.
Resources, such as the stockholding facilities of producers, retailers and
logistics companies will have followed a particular \emph{path of investment}
or development and the actors themselves will have \emph{co-evolved}.

Co-evolution does not refer to an inevitable increase in the ‘closeness’ of
the relationships between interacting actors. Rather, it suggests that
\emph{the operations, characteristics and attitudes of business actors evolve
  as an outcome of their interactions} over time and are affected by the
multiple interaction episodes in their development.

Vargo and Lusch (2004, 2011): "Resources are not: they become”.

\end{frame}

\begin{frame}{Small Worlds in the Wider World}

All the actors are part of a wider network. However, each of these actors has
a very restricted picture of this ‘wider world’ and each has no direct
interaction with the actors within it.

For this reason, each actor will be dependent on \emph{service provision} by
some of its immediate counterparts who have relationships with or provide
access to others at a distance. Using \emph{components off the shelf} (COTS)
not only in a technical but also production-organisational perspective.

For example, the producers of ready-meals are likely to depend on their
relationships with packaging companies to gain access to the activities and
resources of packaging materials companies. Similarly, logistics companies
will depend on their relationships with trailer suppliers to access the skills
of vehicle refrigeration contractors.

\end{frame}

\begin{frame}{Small Worlds in the Wider World}

Access to the wider world is the classic role of \emph{systems integrators}
and \emph{distributors}. The wider world of the business network is made up of
the myriad small worlds of other actors, each of which is comprised of
interlocking interdependencies within which service may be provided and value
created.

This leads to a view of interaction in business relationships as a unique,
evolving, multifaceted process of \emph{‘problem-coping’ by and for all of the
  involved actors} (Webster 1965).

The term ‘coping’ is used to emphasise the interactive and evolving nature of
business problems.

The complex, evolving and interactive nature of problem-coping also mean that
each actor will also have \emph{to conform to the status quo} in many aspects
of its relationships for which problems cannot immediately be addressed.

\end{frame}

\begin{frame}{Business Networking (Cooperation)}\small

Actors are likely to look within their existing relationships when new
problems arise.

Service-seeking and offering drives the process of \emph{activity
  specialisation}, \emph{division of labour by specialisation}, the \emph{path
  of resources} and the \emph{co-evolution of actors}.

The most significant problems that actors face concern \emph{the relationship
  structure in which they are embedded}. The business actor should be viewed
as a node within a network of relationships, so that what happens
\emph{outside} the actor and through its relationships is likely to be more
important in the evolution of that actor than what happens inside.

IMP research uses the term \emph{business networking} to refer to the attempts
of actors to change the structure and process of the relationships in which
they are involved.  It is through business networking that actors seek to cope
with their problems and those of others.

\end{frame}

\begin{frame}{Costs of Business Networking}

Short-term, dyadic problem coping may centre on a single transaction involving
the costs associated with transferring cash for one counterpart and the
benefits of service for the other.

Short-term problem coping may involve working together to solve a particular
technical problem for mutual benefit.

Short-term problem coping may appear to involve only one actor in benefits and
one in only costs. However, these \emph{short-term costs and benefits} received
will affect both actors \emph{long-term view} of their relationship.  The
long-term view considers short-term costs as \emph{investment}.

In the longer term, problem coping will be based on \emph{investments} and
\emph{adaptations} by the counterparts (synergy effects) in one or more
aspects of the substance of their interaction.

\end{frame}

\begin{frame}{Costs of Business Networking}\small

Business actors commonly face issues over the trade-offs between potential and
actual short-term and long-term costs and benefits of the counterparts in
relationships, expressed in terms of the \emph{extent and timing} of
respective activity specialisation, resource path or actor co-evolution.

It is likely that actors would perceive that much of the service actually
provided fell short of their expectations or exceeded them.

Unforeseen contingencies might explain this; the delivery of services or
products may have been late, cooperation may not have been forthcoming,
adaptations may not have been fully carried through, payment may have been
less than expected. In contrast, technical assistance could have produced
greater than anticipated cost savings or a cooperative development could have
enhanced an actor's relationship with a third party.

The existence of different perceptions among actors explains why profitable
business opportunities may exist whenever prices fail to reflect the value.
\end{frame}
  
  
\end{document}
