\documentclass[11pt,a4paper]{article}
\usepackage{ls}
\usepackage[main=english,russian]{babel}
% Requires OpenDiscovery/Pictures (as images) and ls.sty to be linked to the
% current directory.

\setlist{noitemsep}

\title{System Model Innovation with Business TRIZ}

\author{Michelle Bindel, Leipzig Univeersity, Germany}

\date{March 31, 2022}

\begin{document}
\maketitle
\tableofcontents

\section{Introduction}

TRIZ is a contradiction oriented systemic innovation framework that stems from
technology and is constantly being expanded to other fields like Business and
Management. Resulting among other things in Business TRIZ, which applies the
general TRIZ approaches to resolve contradictions to business systems and
business process landscapes. It has already been successfully used to solve
problems in that domain, but there are still challenges to overcome and the
tools are still continuously worked on. The aim of this paper is to explain
the main approaches in Business TRIZ and to put the approaches pursued in
Business TRIZ and its tools and methods in the larger context of innovation
possibilities of business models.  To do that we will look at the development
of Business TRIZ, its tools and how they work together to form a systemically
driven general framework. As well as the current state of business models and
their innovation and how Business TRIZ can be used to innovate them. 

This seminar paper is presented as an extension of a previous talk held in the
seminar \emph{Sustainability, Environment, Management} as part of the module
Applied Computer Science. The seminar is held as a part of the WUMM project
\cite{1}.

\section{From TRIZ to Business TRIZ}

TRIZ is an acronym of the russian phrase "\foreignlanguage{russian}{теория
  решения изобретательских задач}" (Teoria reschenija isobretatjelskich
sadatsch), translating to: theory of inventive problem solving. It consists of
a collection of over 30 tools, which offer a systematic approach for producing
creative ideas. It is founded on the realization that the majority of
inventions complies with a relatively small set of principles and that these
principles can be used to boost our creative thinking. TRIZ does not replace
creativity, but it provides thinking triggers and solution patterns. When
confronted with complex problems, it is not necessary to use tedious
trial-and-error to find a satisfying solution. Instead, it is possible to rely
on TRIZ to be guided to strong and promising approaches. \cite{2}

Initially TRIZ was created by Genrich Saulowitsch Altschuller \cite{3}, a
russian engineer, for other engineers, but within the last 15--20 years its
application was also expanded to non-technical areas. Successful application
to existing (seemingly unsolvable) business problems, triggered the
development of TRIZ for Business and Management, which has still been actively
evolving during recent years. Because TRIZ focuses on studying high-level
patterns and regularities of the non-linear (inventive) evolution of technical
systems, the same or very similar general patterns can be applied to
non-technical systems.

\begin{center}
  \includegraphics[width=.8\textwidth]{images/ProblemSolvingWithTRIZ.png}
    
  \textbf{Figure 1:} The general TRIZ approach to problem solving \cite{4}
\end{center}

In \cite{5} Souchkov identifies several key reasons which make it possible to
apply the principles of TRIZ to non-technical systems and in particular,
business and management systems:
\begin{itemize}
\item Inventive problems emerge due to \emph{conflicts of demands} and none of
  solutions known in an industry where the problem emerged can help.
\item Inventive problems emerge within \emph{systems and processes} and relate
  to organization of and interactions between components of such systems.
\item Challenges and problems are solved through \emph{transforming systems}
  where they emerged.
\item At a high level of abstraction, \emph{patterns of inventive solutions}
  indicating how systems should be transformed appear to be common for most
  diverse types of human-made systems.
\item \emph{General trends} of innovative development (evolution) appear to be
  common for different types of human-made systems which might belong to very
  diverse domains. 
\end{itemize}
The three main approaches in TRIZ are the \emph{solution of contradictions},
the \emph{concept of ideality} and the \emph{trends of system evolution}.
\cite{6}

One of the earliest findings of TRIZ is that most problems are based on a
dilemma or trade-off between two contradicting elements, and that
contradiction has to be solved for a solution. Such a contradiction results
from a conflict in what we want to achieve: either improving a functionality
detoriates another one (\emph{Technical Contradiction}) or the same parameter
has to have opposite values for different functionalities (\emph{Physical
  Contradiction}).

The \emph{Degree of Ideality} is a (qualitative) ratio between the perceived
value delivered by a certain system, product or service and all types of
expenses and investments needed to produce this value. It is used to compare
two competitive systems and is defined as useful functionality of a system
minus all negative factors that diminish its value, divided by costs:
\begin{gather*}
  \text{Degree of Ideality} = \frac{\text{Value Creators}-\text{Value
      Reducers}}{\text{Costs}}
\end{gather*}
If the \emph{Trends of Evolution} are reproducible, then consequently
evolution is not random and TRIZ can be used to predict the future evolution
of certain technologies. Instead of listening to the demands of the customers,
we listen to the "voice" of the product. By knowing which principles form the
basis of the product, we can predict how it will evolve according to the
Trends of Systems Evolution.

\section{TRIZ Tools}

Several tools and techniques were developed by Altshuller and his colleagues
in the advancement of TRIZ. To exemplify some of the more popular tools (that
are represented with current use cases in the \emph{Business TRIZ Online
  Spring 2021 Conference} \cite{7}) are explained here, as it would go beyond
the scope of this paper to name all of them.

\emph{Function Analysis} helps to identify (hidden) interactions within a
system with negative effects or insufficient performance, that may be poorly
controllable, thus uncovering potential for further improvement. It makes it
possible to rank functions delivered by system components and create a
functional hierarchy, while establishing different levels of value delivered
by system components. Valuable functions should be improved and unimportant
ones should be trimmed. An extended version of the \emph{Function Analysis}
was adapted to deal with intangible components such as business decisions or
knowledge and information. \cite{5}

After we have identified the contradictions in a business system, the
\emph{Contradiction Matrix} provides a systematic access to the most relevant
subset of the \emph{40 Inventive Principles}. The columns and rows of the
matrix correspond to the parameters that are affected by the contradiction.
The selected Inventive Principles do not offer an exact solution, but generic
strategies and recommendations that have already successfully resolved similar
contradictions. They still must be translated to a \emph{specific solution},
that can be applied within the context of the specific problem. In the context
of Business TRIZ the \emph{Contradiction Matrix} has been updated by Darrell
Mann in form and content to make it easier for users to connect it with
problems in the business area. \cite{8}

While TRIZ on first glance can appear like a simple toolbox, the variety of
tools can be placed in a systemically driven \emph{general framework} and as a
\emph{uniform process model} they unfold their full potential. Years ago,
Eversheim noted in \cite{9} that the TRIZ methodology does not provide a
strict sequence or specific procedure in the application of the tools.
However, since then, ways of structuring the toolkit have been suggested to
provide some clarity on how the tools should be used.

\begin{center}
  \includegraphics[width=.8\textwidth]{images/qgrbqi.png}
  
  \textbf{Figure 2:} A typical stage-gate process with TRIZ \cite{10}
\end{center}

For example, in \cite{10} Souchkov later formulated a clear
"stage-gate-process", for dealing with every type of innovation task. It
includes four main steps where each step is supported with specific TRIZ tools
adapted for business innovation:
\begin{itemize}
\item \emph{Defining:} The Preparation and Situation Analysis, where goals are
  identified and fixed, revision of demands and constraints is performed and
  project planning is established.
\item \emph{Analysis:} The Problem is analyzed, by using analytical tools to
  structure a situation, build its model, decompose a challenge identified and
  extract key problems which must be solved to meet the project goals.
\item \emph{Generation:} A list (portfolio) of new solution ideas is
  generated.
\item \emph{Selection:} The most promising solutions are ranked and selected,
  and follow-up implementation problems are investigated.
\end{itemize}

\section{Business Model und Business Model Innovation}

Zott, Amit, and Massa note in \cite{11} that it is not universally agreed upon
what a business model is. Through a literature review, among the young and
quite dispersed literature, they show that research on the topic is evolving
in separate silos that are separated by the different interests of the
researchers. Common themes among those are:
\begin{itemize}
\item business models as a new unit of analysis,
\item business models explaining how firms do business, in a system-level and
  holistic approach,
\item business models explaining how value is created and
\item encompassing boundary-spanning activities performed by firms.
\end{itemize}

They propose that by merging these interconnecting and mutually reinforcing
themes business models could be studied in a more unified way. Another
suggestion to advance the study of business models is for researchers to help
others better to understand what a business model in their respective papers
is meant to denote, by employing more precise concepts. This all suggests that
the field is moving toward conceptual consolidation, which they believe is
necessary to pave the way for more cumulative research on business models.

While the definition of business models is hard to pin down, it is necessary
to talk about them in a precise way that makes further research on the topic
possible. Comparable to business models, there is a variety of definitions of
business model innovation and a consensus in the scientific community is still
lacking. But researchers generally agree that the business changes in a way
that results in an increase to the value proposition. In \cite{16} the
following definition was developed as a synthesis of selected definitions from
the literature:
\begin{quote}
  Business model innovation changes the business model in a way that results
  in a \emph{measurable increase in the value proposition}. The innovation of
  business models often does not directly affect the value proposition but is
  usually found in the \emph{key activities} and \emph{key resources}, which
  then influence it.
\end{quote}
Why is business model innovation a relevant topic in academic research? To
understand how it creates and extracts value, every company needs at least one
clearly defined business model and for larger organizations, it is common to
have multiple business units and business models. While business models are
necessary for entrepreneurs starting a company, they are also essential for
established companies wishing to sustain a competitive advantage in the
marketplace. The rapidly developing technology causes constant disruption to
the marketplace and may render existing business models ineffective
anytime. Thus, many authors consider the innovation of business models a key
to competing in the modern economy. And the innovation is necessary at many
levels of the organization and not just at the top.

Generating a new business model and successfully implementing it may come with
a lot of profits. Also a new business model can be expensive and risky and
executing the wrong one can mean insolvency. Depending on its implementation,
the management of business model innovation can provide a firm stability or
uncertainty. But knowing when or how to change a model is difficult. Business
model innovation requires an iterative process and takes art and skill on
behalf of the modeler. Despite the often-experimental nature of creating a
successful business model, there are few dynamic methods for business model
generation. The business landscape is filled with a large range of business
models, but the way new models are generated and innovated are not clear. The
concepts and tools to simplify and understand this environment are missing in
the literature. Consequently, research on business model innovation has
significantly increased.

\section{Business Model according to TRIZ}

In this chapter we show on what definition of a business model TRIZ bases its
approach to business model innovation. In \cite{6} Souchkov forms a conceptual
basis and names structural prerequisites on which the TRIZ tools can operate.
He presents a new approach to business modelling which introduces building
blocks to describe and represent business models.

According to him a \emph{business model} is a broader term than a
\emph{business system} or a \emph{business process}, which are defined and
controlled by the business owners. Additionally, a business model also
encompasses all the components such as external suppliers, customers, and
sometimes even competitors – all parts of a supersystem that is involved in
the process of capturing and delivering value. That is why business models
should be analyzed within a larger context than business systems. In turn,
business process models are context-independent and are used to model flows
and activities arising within business systems and between business systems
and components of their supersystem.

For the Business TRIZ approach to system innovation Souchkov cites the
definition by Johnson, Christensen and Kagermann \cite{12}:
\begin{quote}
  A Business Model is a description of how your company intends to create
  value in the marketplace. It includes unique combination of products,
  services, image, and distribution that your company carries forward. It also
  includes the underlying organization of people, and the operational
  infrastructure that they use to accomplish their work.
\end{quote}
This definition was later structured in \cite{12} into four groups of
components that any business model is comprised of:
\begin{itemize}
\item \emph{Value Proposition:} Value captured and offered by a business
  organization to the market. It can be a technical product, financial
  product, or any type of service.
\item \emph{Profit Formula:} It defines how a business system makes money
  based on delivering its value proposition. In the simplest case, it is the
  "buy low -- sell high" retail formula. Innovative business models introduce
  different variations of approach to sales: lease, monthly payments, credit
  payments, dynamic pricing, and so forth.
\item \emph{Key Activities:} They define main processes and main actions
  needed to create or add value and deliver it to the market.
\item \emph{Key Resources:} These are all kind of resources (labour, capital,
  equipment, etc.) required for successful implementation of key activities.
\end{itemize}

Later Souchkov \cite{6} deems these four components too general and writes
that they miss the business structure. He points to the business model
building blocks defined by Osterwalder and Pigneur \cite{14} as the
\emph{Business Model Canvas}. While still forming the structure of a generic
business model, it is more detailed, but still compact. In their approach, a
business model can be assembled through combining generic building blocks
which specify the way in which the business system operates in detail. In
their model they distinguish between nine building blocks:
\begin{itemize}
\item[1.] \emph{Customer Segments:} An organization serves one or several
  Customer Segments.
\item[2.] \emph{Value Propositions:} It seeks to solve customer problems and
  satisfy customer needs with value propositions.
\item[3.] \emph{Channels:} Value propositions are communicated and delivered
  to customers through communication, distribution, and sales channels.
\item[4.] \emph{Customer Relationships:} Customer relationships are
  established and maintained with each Customer Segment.
\item[5.] \emph{Revenue Streams:} Revenue streams result from value
  propositions successfully offered to customers.
\item[6.] \emph{Key Resources:} Key resources are the assets required to offer
  and deliver the previously described elements.
\item[7.] \emph{Key Activities:} All types of activities needed to perform and
  support the above-mentioned building blocks.
\item[8.] \emph{Key Partnerships:} Some activities are outsourced, and some
  resources are acquired outside the enterprise.
\item[9.] \emph{Cost Structure:} The business model elements result in the
  cost structure.
\end{itemize}
When developing a business model, each of those blocks can be full of a
specific context that is specific to the respective business scenario. It
helps to capture and visualize the most vital and relevant information.

In a business model each of those building blocks is filled with its own
content that is dependent on the specific type of business, service, or
product it represents. Simultaneously each of those blocks can contain a
generic pattern. It is possible to reuse those building block patterns across
different business domains and can be seen as analogue to the physical
principles in technical TRIZ. So, when the need to design a new business model
or improve an existing one arrives, it is possible to choose the most fitting
pattern or batch of \emph{patterns from a predefined pattern database}. One
such database does currently not exist but is a topic of further research in
the field of Business TRIZ.

On that basis, we see that innovation is possible in two different ways. One,
by changing the content of one or more of those building blocks, to improve
the business model in a mild or radical manner. And second by building a
completely new business model out of several building blocks.

\section{Business Model Innovation in Business TRIZ}
\begin{center}
  \includegraphics[width=.8\textwidth]{images/TWIt7w.png}
  
  \textbf{Figure 3:} Business Model innovation as an important part of
  Business Innovation \cite{5}
\end{center}

The ever-changing business environment requires continuous innovation of the
ways in which we practice business and the technology involved in it. Random
methods of idea generation are not efficient enough to keep pace. That is why
the focus rests on finding systematic methods that support a continuous
process of new business ideas generation. Since TRIZ is a leading systematic
discipline for supporting the early stages of innovation, it is only logical
to explore the applicability of TRIZ to business model innovation as well.

It is important to distinguish the term business model innovation from the
whole of business innovation. While Business model innovation has become very
popular and is affecting all other types of business innovation it is only a
part of it. Souchkov \cite{10} differentiates between them by saying "when we
consider business system/network innovation we focus on the components change
while when we focus on business model innovation, we primarily change the
relationships between the components of a business system and its
supersystem." In \cite{10} Souchkov compiles a list of examples for typical
tasks in business model innovation and describes the possibilities of the
application of key TRIZ tools and concepts such as Ideality within the context
of business models.


\begin{center}
  \includegraphics[width=.8\textwidth]{images/77uMRr.png}
    
  \textbf{Figure 4:} Typical innovative tasks for business model innovation
\end{center}

An important role in the application of TRIZ to business models plays the
concept of \emph{business building blocks} explained in section 5, that helps
translate the TRIZ methods to the components of the business model. They
describe business models clearly and in a structured way without overloading
it with numerous details and make it possible to systematically assess and
analyze business models with the TRIZ analytical tools and to innovatively
modify existing business models or to design new, innovative ones. They help
to locate and define problems, contradictions, and areas with high evolution
potential. \cite{6}

Business models tend to evolve according to the generic TRIZ trend of
Increasing degree of Ideality. The Ideality formula as introduced in
section 2 applies to both business and technical systems where:
\begin{itemize}
\item \emph{Value Creators} are all parameters, useful features and functions
  of a Value Proposition (product or service) which are positively perceived
  by the market,
\item \emph{Value Reducers} are those features, functions, harms and any other
  factors that reduce the perceived value and
\item \emph{Costs} are all direct and indirect expenses required to generate
  and maintain the Value Creators.
\end{itemize}

\begin{center}
  \includegraphics[width=.8\textwidth]{images/YNlty3.png}
    
  \textbf{Figure 5:} A selection of common phases of innovative projects,\\
  relevant for business model innovation and the TRIZ tool supporting these
  \cite{5}
\end{center}

The higher the degree of Ideality of a specific Value Proposition within a
certain segment is, the more competitive the Value Proposition will be. In
turn, the degree of Ideality of a specific Value Proposition depends on the
degree of Ideality of a business model used to create the Value
Proposition. To increase the degree of Ideality of any business model it is
possible to innovate in all three categories: Increase Value Creators by
increasing the perceived value of the offerings in many different
ways. Eliminate Value Reducers by working against all harmful effects that
negatively affect the perceived value of the offering. Eliminate or Reduce
Costs without both decreasing Value Creators and increasing Value Reducers.
While it was claimed in earlier works that Typical Patterns of Business Model
Innovation are identical to the patterns known for technical systems and
already represented in the TRIZ knowledge bank, that has proven only partly
true. Studies show that there are also several unique patterns specific to the
business domain. Similar to technology and engineering, there are a number of
universal high-order patterns of solution strategies which resolve
contradictions and overcome barriers created by solutions known within the
system’s domain. In \cite{15} it was tried to structure such patterns and
apply them to exemplary scenarios of business model innovation to demonstrate
their effectiveness.

Next to the direct approaches to use the TRIZ principles in business model
innovation, over the years there have also been successful attempts to combine
them with other frameworks that try to foster innovation in business. In
\cite{17} a business model innovation model based on the respective advantages
of \emph{Case-Based Reasoning} (CBR) and TRIZ is proposed. A structured
framework for setting up a comprehensive \emph{SWOT analysis} integrating
TRIZ-based tools is introduced in \cite{18} that is supposed to help to define
high reward areas of innovation in the early phase of planning processes. In
\cite{19} TRIZ tools are shown to contribute to the \emph{Lean Canvas
  Analysis} and helping especially start-ups to accelerate a methodical
innovation process of their business model.

\section{Discussion and Conclusion}

It has been proven multiple times that the adaption of TRIZ to Business TRIZ
was successful. Souchkov reports about many promising experiences with the use
of TRIZ to deal with business and management innovative challenges from
different countries \cite{5}. While the introduction of TRIZ was rather slow,
compared to methods that are backed by less extensive frameworks, it is still
relevant and has not vanished into insignificance. Old hurdles to TRIZ are
still relevant \cite{20}, as it needs a significant amount of studying before
it can be successfully applied and the original technical domains it evolved
in having a different language and concepts than the business audience is used
to.

Through the nature of the fast-moving field of business and management there
are constantly new challenges emerging and new fields of study, like business
model innovation, opening up. And that TRIZ is being applied to these new
problems by researchers, shows its continued relevance in the innovation
sciences. The development of Business TRIZ still proceeds and like the
business landscape the methods and tools of TRIZ always need to be adapted to
the current needs. There are already many successful cases of that
(reformulation of inventive principles, inventive standards, ARIZ). It can
also be observed that it doesn’t involve inside a bubble but is being put in
combination with other current approaches, which shows its versatility.

The innovation of business models is one such exemplary area in which the work
through TRIZ has been welcome and helpful but might never finished, as the
constantly changing nature of the business world dictates.

\begin{thebibliography}{xxx}
\bibitem{3} Altshuller, G. S. (1984). Creativity as an exact science: the
  theory of the solution of inventive problems. Gordon and Breach.
\bibitem{18} Brad, S., Brad, E. (2015). Enhancing SWOT analysis with
  TRIZ-based tools to integrate systematic innovation in early task design.
  \emph{Procedia engineering}, 131, pp. 616--625.
\bibitem{7} Results of the \emph{Business TRIZ Online Conference Spring 2021}.
  See \url{http://wumm.uni-leipzig.de/conferences.php}
\bibitem{9} Eversheim, W. (ed., 2008). Innovation management for technical
  products: systematic and integrated product development and production
  planning. Springer Science \& Business Media.
\bibitem{15} Gibson, E., Jetter, A. (2014). Towards a dynamic process for
  business model innovation: A review of the state-of-the-art.
  \emph{Proceedings of PICMET'14 Conference}, pp. 1230--1238.  IEEE.
\bibitem{20} Ilevbare, I. M., Probert, D., Phaal, R. (2013). A review of TRIZ,
  and its benefits and challenges in practice. \emph{Technovation}, 33 (2-3),
  pp. 30--37.
\bibitem{12} Johnson, M. W., Christensen, C. M., Kagermann, H. (2008).
  Reinventing your business model. \emph{Harvard Business Review}, 86 (12),
  pp. 57--68.
\bibitem{8} Mann, D. (2005). New and emerging contradiction elimination tools.
  \emph{Creativity and Innovation Management}, 14 (1), pp. 14--21.
\bibitem{14} Osterwalder, A., Pigneur, Y. (2010). Business Model Generation: a
  Handbook for Visionaries, Game Changers, and Challengers (vol. 1). John
  Wiley \& Sons.
\bibitem{17} Shao, M., Ding, J., Ding, M., Liu, X. (2012). Research on
  business model innovation method based on TRIZ and CBR. \emph{Proceedings of
    the Second International Conference on Business Computing and Global
    Informatization}, pp. 895--898. IEEE.
\bibitem{19} Sire, P., Prevost, E., Guillou, Y., Riwan, A., Saulais, P.
  (2019).  How can TRIZ tools tremendously stimulate the Lean canvas analysis
  to foster start-up business model and value proposition? \emph{International
    TRIZ Future Conference}, pp. 93--105. Springer.
\bibitem{4} Souchkov, V. (2007). Breakthrough thinking with TRIZ for business
  and management: An overview. ICG Training \& Consulting, pp. 3--12.
\bibitem{6} Souchkov, V. (2010). TRIZ and systematic business model
  innovation.  In Global ETRIA Conference TRIZ Future, pp. 3--5.
\bibitem{16} Souchkov, V. (2015). Typical Patterns of Business Model
  Innovation. \emph{The Journal of the European TRIZ Association}, Special
  Issue: Collection of papers of 15th International TRIZ Future
  Conference-Global Structured Innovation,  pp. 121--127.
\bibitem{2} Souchkov, V. (2017). A short introduction to TRIZ and xTRIZ.
  \url{https://www.youtube.com/watch?v=XRsyyGPXNA4}
\bibitem{13} Souchkov, V. (2007-2019).  Training Manual for Systematic
  Business Innovation.  ICG Training \& Consulting.
\bibitem{10} Souchkov, V. (2019). Systematic business innovation: a roadmap.
  \emph{TRIZ Review}, pp. 122-132.
\bibitem{5} Souchkov, V. (2019). TRIZ for business and management: State of
  the art. Proceedings of the TRIZ Developers Summit.
\bibitem{1} The WUMM Project. \url{https://wumm-project.github.io}.
\bibitem{11} Zott, C., Amit, R., Massa, L. (2011). The Business Model: Recent
  Developments and Future Research. \emph{Journal of Management}, 37 (4),
  pp. 1019–1042.
\end{thebibliography}
\end{document}
