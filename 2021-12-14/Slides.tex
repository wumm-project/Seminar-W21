\documentclass{beamer}
\usepackage{lsfolien}
\usepackage[english]{babel}

\myfootline{Sustainability, Environment, Management -- Winter Term
  2021}{Hans-Gert Gr\"abe}

\newcommand{\ueberschrift}[1]{\begin{center}\bf #1\end{center}}

\title{Exploitative and Explorative\\ Business Process Improvement \vskip1em}

\subtitle{Research Seminar in the Module 10-202-2309\\ for Master Computer
  Science}

\author{Prof. Dr. Hans-Gert Gräbe\\
\url{http://www.informatik.uni-leipzig.de/~graebe}}

\date{December 2021}
\begin{document}

{\setbeamertemplate{footline}{}
\begin{frame}
  \titlepage
\end{frame}}

\begin{frame}{Some Conceptual Points}

We are faced in the sessions of our seminar with several concepts related
to Business Processes:
\begin{itemize}
\item BP Modelling
\item BP Landscaping
\item BP Execution
\item BP Management
\item BP Improvement
\end{itemize}

What ist the relation between these concepts? How they are contextualised?
What ist the relation to system concepts?

\end{frame}

\begin{frame}{Business Process Management}

(Lindskog 2018):
\begin{itemize}
\item BP Management (BPM) = the body of methods, techniques, and tools to
  discover, analyze, redesign, execute and monitor business processes.
\item BPM has traditionally focused on increasing the efficiency and
  effectiveness of business processes through exploitation, standardization or
  automation.
\item 60–80\% of BPM initiatives have been fruitless.
\item Unlike \emph{exploitation}, which is driven by current practices,
  \emph{exploration} is focused on possible future process practices.
\end{itemize}

In this sense, BPM is the management of the feedback loop between the \emph{BP
  model} and the \emph{real processes}.
\end{frame}
\begin{frame}{BPM Lifecycle}

BPM is the process of alignment between model and reality, which is usually
\emph{modelled as BPM lifecycle} (as a \emph{model} of alignment of model and
reality).
  \begin{center}
    \begin{minipage}[b]{.45\textwidth}\centering
      \includegraphics[width=\textwidth]{images/Hills-BP-lifecycle-40.png}\\
      (Meidan et al. 2016)
    \end{minipage}
    \hfill 
    \begin{minipage}[b]{.45\textwidth}\centering
      \includegraphics[width=\textwidth]{images/Dumas-BPM-lifecycle.png}\\
      (Meroni 2018)
    \end{minipage}
  \end{center}
\end{frame}

\begin{frame}{BPM Lifecycle} 
  \begin{center}
    \begin{minipage}[b]{.45\textwidth}\centering
      There exist different models of that lifecycle.\vskip3em

      \includegraphics[width=\textwidth]{images/Business-process-life-cycle.png}\\
      (Ruiz et al. 2012)
    \end{minipage}
    \hfill 
    \begin{minipage}[b]{.45\textwidth}\centering
      \includegraphics[width=\textwidth]{images/3-Figure1-1.png}\\
      (Muehlen, Ho 2005)
    \end{minipage}
  \end{center}
  
\end{frame}

\begin{frame}{Business Process Improvement}

  Business Process Improvement (BPI) thus corresponds to the systemic
  developmental step from the \emph{System as is} to the \emph{System as
    required} in its dialectical contradictoriness, as explained in more
  detail in the lecture.

  Here, too, a distinction must be made between the \textbf{real
    transformation step} and the \textbf{modelling of this step}.

\end{frame}

\begin{frame}{Business Process Improvement}

  (Rosemann 2020)

  \textbf{Exploitative process improvement} is dedicated to process efficiency
  and effectiveness within a given value proposition. Related approaches are
  either comprehensive methodologies (e.g., Lean Management, Six Sigma, Theory
  of Constraints, Business Process Reengineering) or sets of fine granular,
  operational improvement heuristics.

  The former tend to concentrate on the issue identification and are either
  light and unstructured (e.g., brainstorming) or very narrow (e.g., how to
  overcome variation) in terms of suggesting actual improvements.
\end{frame}

\begin{frame}{Explorative Process Improvement}

  The latter are grounded in the tradition of TRIZ, i.e. the theory of the
  resolution of invention-related tasks, which proposed a set of generalizable
  solution patterns. Examples for such process improvement patterns are
  elimination (delete an activity), integration (merge two activities),
  automation (of an activity) or optionality (make an activity optional for
  some process stakeholders).
  
  \textbf{Explorative process improvement:} An operationally efficient process
  can become obsolete when the value proposition of the business process is
  threatened and the process itself ultimately could become nonrelevant.

  \textbf{Revenue resilience} (as a \textbf{design paradigm} -- Rosemann)
  describes the capability to sufficiently defend and grow the revenue base in
  light of disruptive forces.

\end{frame}

\begin{frame}{Explorative Process Improvement}
  
  The aim of explorative BPM, the search for new value, is comparable to the
  \textbf{intentions of a business model}, i.e. a description of how an
  organization creates, captures and monetizes value.

  Explorative process design patterns complement a business model by 
  \begin{itemize}
  \item[(1)] providing an explicit view on processes as a source of new value,
    and 
  \item[(2)] shifting the dominating focus of exploitative BPM, which
    concentrates on \textbf{cost structures}, to the study of how business
    process explorations can lead to \textbf{new revenue streams}.
  \end{itemize}

\end{frame}

\begin{frame}{Explorative Process Improvement}
  
  Rosemann proposes seven Explorative Process Improvement Design Patterns
  \begin{itemize}
  \item Process Generalisation
  \item Process Expansion
  \item Process Differentiation
  \item Process Initiation
  \item Process Commercialisation
  \item Process Integration
  \item Process Attention
  \end{itemize}
  
\end{frame}
  
\end{document}
