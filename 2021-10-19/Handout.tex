\documentclass[11pt,a4paper]{article}
\usepackage{ls}
\usepackage[english]{babel}

\setlist{noitemsep}

\title{On the Notion of a System}  

\newcommand{\Merksatz}[1]{
  \begin{center}
    \fbox{\fboxsep12pt \fbox{\parbox{.9\textwidth}{\textbf{Remember:} #1}}}
  \end{center}
}

\author{Hans-Gert Gr\"abe}

\date{October 18, 2021}

\begin{document}
\maketitle

\section{Once more about the Goal of the Seminar}

Systematic innovation methodologies such as TRIZ are essentially based on a
better understanding of the development dynamics of corresponding (technical
and non-technical) \emph{systems}.  The results are rooted in engineering
experience from structured processes of planning, implementation and operation
of technical systems. Increasingly, cooperative interdisciplinary
collaboration matters rather than the one brilliant mind that commands
thousand hands. The \emph{socio-technical character} of contradictions is
thereby intensified and opens up new dimensions of contradiction management.

Today, managers face similar challenges when it comes to placing
decision-making processes on a systematic basis, aligning the processes under
control with long term goals, and also achieving the targeted goal corridors.
It turns out that many engineering experiences on structured procedures in
contradictory requirement situations can be transferred to this area, which
has been investigated within the topic "TRIZ and Business" for 20 years.

Nevertheless, experiences and approaches to theories of systemic concepts are
based more broadly and also have much longer historical traditions.  \emph{In
  the seminar}, we want to study such concepts more closely, with special
emphasis on cooperative approaches in interdisciplinary contexts.

\section{What is a System?}

Operation and use of technical systems is a central element of today world
changing human practices. For this purpose planned and coordinated action
along a division of labour is necessary, because exploiting the benefit of a
system requires its operation. Conversely, it makes little sense to operate a
system that is not being used. Closely related to this distinction between
definition and call of a function, well known from computer science, is the
distinction between design time and runtime, that is even more important in
the real-world use of technical systems – during design time, the principal
cooperative interaction is planned, during the runtime the plan is
executed. For technical systems one has to distinguish the description,
interpersonally communicated as justified expectations, and the results of
operation, interpersonally communicated as practical experience.

Most definitions grasp the term \emph{system} as a delimited set of
interacting components, whereby the interaction of the components gives rise
to a unified whole, which realises an emergent function and is thus more than
the sum of its parts.
\begin{quote}
  A \emph{system} (lat. greek "system", "composed", a whole consisting of
  parts; connection) is a set of elements that are interconnected and interact
  with each other, forming a unified whole that possesses properties that are
  not already contained in the constituent elements considered individually.
  \cite{Petrov2020}

  A \emph{system} is a set of elements that are in relationship and connection
  with each other and that constitute a well defined unity, an integrity. The
  necessity of the use of the term "system" occurs when it is required to
  emphasize that something is large, complex, immediately not wholly
  comprehensible, but at the same time a unified whole. Unlike the notions
  "set" or "aggregate", the concept of a system emphasizes the ordering, the
  integrity, the regularity of construction, functioning and development.
  \cite{TOP}

  Systems Engineering "is an interdisciplinary field of engineering and
  engineering management that focuses on how to design, integrate, and manage
  complex systems over their life cycles. At its core, systems engineering
  utilizes systems thinking principles to organize this body of knowledge. The
  individual outcome of such efforts, an engineered system, can be defined as
  a combination of components that work in synergy to collectively perform a
  useful function." (English Wikipedia).
\end{quote}
The second definition also points to the purpose of systemic delimitations --
it is about making complex relationships accessible to description by reducing
them to essentials.

In all these definitions, the \emph{structuredness} and thus
\emph{decomposability} of the system in the analytic dimension is emphasised
on the one hand, and the \emph{interdependence} and thus
\emph{indecomposability} in the execution dimension on the other. This
corresponds to the practical experience of the engineer when she assembles a
system from individual components -- the system is only viable when it is
assembled. In the assembled system in addition to the components, the
\emph{connecting elements} also play an important role.  They mediate the
\emph{flow of energy, material and information} that is required for the
operation of each component. In component software \cite{Szyperski2002}, with
\emph{deployment}, \emph{installation} and \emph{configuration} three stages
of preparing components for their operation in a systemic context are
distinguished, and this preparation for operation is often considered as a
separate state -- for example, as \emph{maintenance mode} different from the
\emph{operation mode}.

The aspect of operating a system did not play a role in the first two
definitions.  Only here, however, the dialectical interrelationship between
decomposability and indecomposability comes to light: Viable components
deliver processual services in \emph{guaranteed} quantity and quality during
operation, if the \emph{external operational conditions} are guaranteed.
These processual services of the components in combination form the emergent
function of the overall system. The self-similarity of the concept is obvious:
components themselves have an inner life that can be described systemically,
but which is largely abstracted from at the level of the overall system. The
component enters the description of the dynamics of the overall system only as
Black Box with a precise specification.  This specification is divided into
input and output interfaces.  The former describe the necessary operating
conditions, the latter the performance parameters of the respective component.

In \cite{Graebe2020} the system concept is identified as descriptional
focusing to make real-world phenomena accessible for a description by
reduction to the essentials.  Such a reduction focuses on the following three
dimensions:
\begin{itemize}
\item[(1)] Outer demarcation of the system against an \emph{environment},
  reduction of these relationships to input/output relationships and
  guaranteed throughput.
\item[(2)] Inner demarcation of the system by combining subareas to
  \emph{components}, whose functioning is reduced to “behavioural control” via
  input/output relations.
\item[(3)] Reduction of the relations in the system itself to “causally
  essential” relationships.
\end{itemize}
Further, it is stated that such a reductive description (explicitly or
implicitly) exploits output from prior life:
\begin{itemize}
\item[(1)] An at least vague idea about the (working) input/output services of
  the environment.
\item[(2)] A clear idea of the inner workings of the components (beyond the
  pure specification).
\item[(3)] An at least vague idea about causalities in the system itself, that
  precedes the detailed modelling.
\end{itemize}

\section{Systems, Components and Reuse}

One important aspect, especially of technical systems, has not yet been taken
into account in the considerations so far: the aspect of reuse. Reuse plays a
central role in computer science -- copy/paste of code, outsourcing of
repetitive pieces of code in function definitions, grouping of related
function definitions in pre-compiled libraries, etc. This in no way exhausts
possible forms of reuse, not to mention higher forms of reuse such as design,
patterns or frameworks. Szyperski discusses in \cite[ch. 8]{Szyperski2002}
aspects of the relationship between goals and forms of reuse.

Hence in addition to the description and operation, for technical systems the
aspect of reuse plays an important role. However, this does not apply, at
least on the artifact level, to larger technical systems – these are unique
specimen, even though assembled using standardised components. Also the
majority of computer scientists is concerned with the creation of such unique
specimens, because the IT systems that control such plants are also unique.

Computer science has long struggled with a form of reuse that is widespread in
developed engineering sciences and ultimately turned the manufacturing of
tools and products from an art first to a craft and later to an industrial
process -- the \emph{use of components produced by third parties} (components
off the shelf).

Thus, after the analytical and operational dimension of systems and
components, the \emph{production by independent third parties} and hence the
technical-economic interrelationships of an industrial mode of production
based on the division of labour move into the focus of attention.

In such a context, the concept of a technical system is fourfold overloaded.
A technical systems can be considered
\begin{itemize}
\item [1.] as a real-world unique specimen (e.g. as a product or a service),
\item [2.] as a description of this real-world unique specimen (e.g. in the
  form of a special product configuration)
\end{itemize}
and for components produced in larger quantities also
\newpage
\begin{itemize}
\item [3.] as description of the design of the system template (product
  design) and
\item [4.] as description and operation of the delivery and operating
  structures of the real-world unique specimens of this system produced
  according to this template (as production, quality assurance, delivery,
  operational and maintenance plans).
\end{itemize}
The concept of a technical system thus has also in this context a clearly
epistemic function of (functional) ``reduction to the essential''. To Einstein
the recommendation is attributed ``to make things as simple as possible but
not simpler''. The TRIZ \emph{law of completeness of a system} expresses
exactly this thought, however, not as a \emph{law}, but as an engineering
\emph{modeling directive}. The apparent ``law'' of the observed dynamics
therefore essentially addresses \emph{reasonable human action}.

In an approach of ``reduction to the essential'' and ``guaranteed
specification-compliant operation'' human practices are inherently built in,
since only in such a context the terms ``essential'', ``guarantee'' and
``operation'' can be filled with sense in a meaningful way. These essential
terms from the socially determined practical relationship of people are deeply
rooted in the concept generation processes of descriptions of special
technical systems and find their ``natural'' continuation in the special
social settings of a legally constituted societal system.

\section{Socio-technical Systems}

The last considerations already embed the concept of system in social
practices of cooperatively acting people. This embedding is also present in
TRIZ system definitions, when the emergent function realised by the system is
considered as \emph{main useful function} MUF and linked to a \emph{purpose},
why this (technical) system exists or was designed or redesigned in this way.
This \emph{aspect of purposefulness} (Zweckmäßigkeit) plays only a subordinate
role for "natural" systems, namely for socio-ecological systems, since in this
context in most cases the "purposefulness" comes up against hard limits or
causes massive problems or has even already caused them. Nevertheless, this
\emph{orientation on purposes} is another throughput parameter (e.g. as
monetary throughput) from a social environment relevant for the inner dynamics
of a system.  It can ultimately be subsumed under the \emph{throughput of
  information} if a sufficiently viable concept of information is taken as a
basis. 

This purposefulness transforms the totality of technical systems into an
interconnected \emph{world of techical systems} full of preconditions and
conditionalities, which opens up a fourth dimension of the concept of system,
to secure stable operating conditions of the systems themselves.

The self-similarity of the systems concept provides a solution for this
challenge -- consider systems as components and the relations of
purposefulness as interdependencies, delineate larger socio-technical systemic
units, develop appropriate forms of description and operation. The
transformation towards a sustainable mode of production and living that is on
the agenda just requires a big step forward in this direction.  This is one of
the objectives of management and hence in the primary focus of our seminar.
However, socio-technical systems are, in addition to technical restrictions,
charged with the contradictory expectations and interests of concrete people
and groups of people.

Ian Sommerville \cite{Sommerville2015} elaborates a number of challenges in
this regard. He also starts with the concept of a goal-centered system.
\begin{quote}
  A \emph{system} is a meaningful set of interconnected components that work
  together to achieve a specific goal.  \cite{Sommerville2015}
\end{quote}
Right after he develops a distinction between technical and socio-technical
systems:

\paragraph{Technical computer-based systems}
are systems that contain hardware and software components, but not procedures
and processes. ... Individuals and organisations use technical systems for
specific purposes, but knowledge of that purpose is not part of the system.
For example, the word processor I use does not know that I am using it to
write a book.

\paragraph{Socio-technical systems}
contain one or more technical systems, but beyond that -- and this is crucial
-- the knowledge of how the system should be used to achieve a broader
purpose.  This means that these systems have \emph{defined work processes},
\emph{human operators} as integral part of the system, are \emph{governed by
  organisational policies} and are \emph{affected by external constraints}
such as national laws and regulations.

Essential characteristics of socio-technical systems:
\begin{enumerate}
\item They have special properties that affect the system as a whole, and are
  not related to individual parts of the system. These special properties
  depend on the system components and the relationships between them. Because
  of this complexity, the system-specific properties can only be evaluated
  when the system is composed.
\item They are often not deterministic. The behaviour of the system depends on
  the human operators and on other people who do not always react in the same
  way. Also, the operation of the system can change the system itself.
\item The extent to which the system supports organisational goals depends not
  only on the system itself. It also depends on the \emph{stability of the
    goals}, the relationships and \emph{conflicts between organisational
    goals}, and how people in the organisation \emph{interpret those goals}.
\end{enumerate}

In this context, there is a clear shift on the scale of controllability from
direct control by external human operators to indirect control and movement
according to intrinsic laws, which is even more prevalent in
\textbf{socio-economic systems} with a large number of stakeholders or even
\textbf{socio-ecological systems}.

\section{Shchedrovitsky on Systems Analysis}

The system concept thus serves to delimit a part of the complex, all-connected
world (hereafter \emph{reality}) in order to make this part accessible for
description. However, this human activity, which Georgi Shchedrovitsky (a
Russian Philosopher and the head of the \emph{Methodological School of
  Management}) refers to as \emph{mental activity} \cite[p. 47]{MSM}, is
itself part of that reality and thus also of practical relevance. Real-world
processes are thereby charged with description forms. Thus in systems these
two dimensions -- description and operation -- must therefore be
distinguished. Charging a system with a description form is what Engels'
calls, in reference to Kant's \emph{thing in itself}, the transformation of
the \emph{thing in itself} into a \emph{thing for us}.

Shchedrovitsky \cite[p. 80 ff.]{MSM} conceptualises this process in two
different concepts of system \cite[pp. 89 and 98]{MSM} as process of breaking
down the system into parts (components), charging the components with
description forms and then reassembling the components thus charged into a
whole.  The result is a \emph{new} system in the sense that it is the old one
but charged with a description form.  In this way, the \emph{structural
  organisation} of a system can be grasped.

The real world and thus also systems develop and change over time. In order to
understand the \emph{development of a system}, its \emph{processual
  organisation} must be examined.  Shchedrovitsky emphasises that the
development of a system can \emph{never} be described in its disassembled
form, since disassembly destroys the systemic coherence. An aeroplane
disassembled into its individual parts cannot fly, only an assembled one. We
are dealing here with a fundamental epistemic contradiction.

For details we refer to \cite{MSM}.

\section{Theory of Dynamical Systems}

\subsection{The Approach}

The processual dimension of systems can be investigated with the mathematical
tools of the Theory of Dynamical Systems if the processes can be modelled as
equations of motion in phase space.

The Theory of Dynamical Systems as a branch of mathematics investigates the
dynamics of structurally defined and modelled systems. Attributes which are
essential for the description of the system are combined into a \emph{phase
  space} and the changes in the attribute values are described as
\emph{equations of motion} by differential equations. If only temporal changes
are considered, this leads to systems of \emph{Ordinary Differential
  Equations} (ODE), complex spatio-temporal changes lead to Partial
Differential Equations. We restrict ourselves to the first case, i.e. purely
temporal structural changes.

In the simplest case, such as the pendulum or the movement of two bodies in a
homogeneous gravitaional field, a \emph{trajectory} can be calculated from the
equations of motion.

Examples:
\begin{itemize}
\item Pendulum: \url{https://en.wikipedia.org/wiki/Pendulum_(mechanics)}
\item Two body problem: \url{https://en.wikipedia.org/wiki/Two-body_problem}
\end{itemize}

\subsection{Model and Reality}

However, the solution of this equations only describes the motion $m(t)$ in
the model. Good modelling is characterised by the fact that the real movement
$f(t)$ and the movement $m(t)$ according to the model differ only
insignificantly $r(t)=f(t)-m(t)$ in practically relevant parameter ranges (the
\emph{context of observation}). This can only be verified empirically through
experiments that are to be planned more or less precisely, since reality is
only accessible empirically.

Particularly interesting are modellings in which the residual $r(t)$ decreases
"by itself". Such systems strive towards an equilibrium, which structure can
be derived from the model.

\subsection{How Chaotic can Trajectories be?} 

Examples: 
\begin{itemize}
\item Double Pendulum, \url{https://en.wikipedia.org/wiki/Double_pendulum}
\item Magnetic pendulum with three attracting magnets,
\item 3-body model: \url{https://en.wikipedia.org/wiki/Three-body_problem}
\end{itemize}

We see that there is apparent stability for a long time, but in phase space
there are certain \emph{areas of instability} in which (exactly calculable!)
trajectories passing through points in phase space that are close to each
other strongly diverge. Such locations are called \emph{bifurcations}.  Often
there is a single phase parameter that makes this bifurcation particularly
clear. Such a bifurcation on a one-dimensional scale is also called a
\emph{tipping point}.

Not everything that looks like chaos has to be chaotic:\\
\url{https://i.redd.it/zr7tet9mdfl01.gif}

\subsection{Attractors}

How complicated can an equilibrium position be? 

Examples: 
\begin{itemize}
\item Pendulum,
\item pendulum with three attracting magnets,
\item pendulum with one repelling magnet.
\end{itemize}

Limit cycles: \url{https://en.wikipedia.org/wiki/Limit_cycle}

When the body is on the limit cycle, it remains there, i.e. the limit cycle is
a \emph{stable solution} of the equations of motion of the system, called
\textbf{steady-state equilibrium}.
    
In many cases the real movement $f(t)$ in time is \emph{attracted} by that
limit cycle, i.e. $f(t)$ can be decomposed into $f(t)=l(t)+r(t)$ with $l(t)$
the projection on the limit cycle and $r(t)$ a (small) orthogonal deviation.
In this way, it is often possible to simplify complicated models.

An attractor is a specific steady-state equilibrium with just this attracting
property.

More precisely: Let $f(t,a)$ be a function which specifies the dynamics of the
system with starting point $f(0,a)=a$. An \textbf{attractor} is a subset $A$
of the phase space characterized by the following three conditions:
\begin{itemize}
\item $A$ is forward invariant under $f$: if $a$ is an element of $A$ then so
  is $f(t,a)$, for all $t > 0$.  
\item There exists a neighborhood of $A$, called the basin of attraction for $A$
  and denoted $B(A)$, which consists of all points $b$ that "enter $A$ in the limit
  $t\to\infty$". 
\item There is no proper (non-empty) subset of $A$ having the first two
  properties.
\end{itemize}

Attractor as stable solution of the corresponding system of ODE\\
\url{https://en.wikipedia.org/wiki/Attractor}

\paragraph{On the importance of "stable" cyclical processes in nature.}
We are able to perceive such \emph{approximately} repeating patterns in
natural processes (i.e. attractors), i.e.  perform such a reduction also
independently of mathematical abilities.

For given (deterministic) equations of motion one can compute the geometry of
such an attractor as \emph{global deterministic} invariant of the equations of
motion.

\paragraph{How Complicated can an Attractor be?}
\begin{itemize}
\item \url{https://en.wikipedia.org/wiki/Attractor}
\item \url{https://en.wikipedia.org/wiki/Lorenz_system}
\item \url{https://de.wikipedia.org/wiki/Lorenz-Attraktor}
\item Attention, with the numerical methods used there for visualisation it is
  difficult to distinguish whether they are calculating a chaotic trajectory
  or really the attractor, which is a \emph{global} artefact.
\item "Almost all initial points will tend to an invariant set – the Lorenz
  attractor – a strange attractor, a fractal, and a self-excited attractor"
  (Wikipedia)
\end{itemize}

\subsection{Dissipative Systems}


Closed and Open Systems. Previous investigations were directed towards the
inner dynamics of an autonomous, i.e. closed system.

Importance of a (stable) throughput of energy, matter and information for the
inner structure formation in systems.

\begin{itemize}
\item Self-organisation in dissipative structures
  \begin{itemize}
  \item \url{https://en.wikipedia.org/wiki/Rayleigh-Bénard_convection}
  \item \url{https://en.wikipedia.org/wiki/Belousov-Zhabotinsky_reaction}
  \end{itemize}
\item Dissipative systems
  \url{https://en.wikipedia.org/wiki/Dissipative_system} 
\item Life on Earth as a dissipative system.
\end{itemize}

\begin{thebibliography}{xxx}
\bibitem{Graebe2020} Hans-Gert Gräbe (2020). Men and their technical systems
  (in German).\\ LIFIS Online, 19 May 2020.
  \url{https://doi.org/10.14625/graebe_20200519}.

  A shorter English version is available at\\
  \url{https://hg-graebe.de/EigeneTexte/sys-20-en.pdf}
\bibitem{Petrov2020} Vladimir Petrov (2020). Laws and patterns of systems
  development (in Russian). ISBN 978-5-0051-5728-7.
\bibitem{MSM} Georgy P. Shchedrovitsky. Selected Works. Part I in Viktor
  B. Khristenko, Andrei G. Reus, Alexander P. Zinchenko et al. (2014).
  Methodological School of Management. Bloomsbury Publishing.  ISBN
  978-1-4729-1029-5.
\bibitem{Sommerville2015} Ian Sommerville (2015). Software Engineering.
  Chapter 19 „Systems Engineering“.

  Slide stack available at Sildeshare\\{\small
  \url{https://www.slideshare.net/software-engineering-book/ch19-systems-engineering}}
\bibitem{Szyperski2002} Clemens Szyperski (2002). Component Software. Pearson
  Education.  2. Auf\-lage.  ISBN 0201745720.
\bibitem{TOP} The TRIZ Ontology Project.
  \url{https://wumm-project.github.io/Ontology.html}.
\end{thebibliography}

\end{document}

