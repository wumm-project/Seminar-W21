\documentclass{beamer}
\usepackage{lsfolien,enumitem}
\usepackage[english]{babel}
\setlist[itemize]{noitemsep,label={\color{blue}$\rhd$}}
           
\myfootline{Sustainability, Environment, Management -- Winter Term
  2021}{Hans-Gert Gr\"abe}

\newcommand{\ueberschrift}[1]{\begin{center}\bf #1\end{center}}

\parskip1em

\title{On the Notion of a System}

\subtitle{Research Seminar in the Module 10-202-2309\\ for Master Computer
  Science}

\author{Prof. Dr. Hans-Gert Gräbe\\
\url{http://www.informatik.uni-leipzig.de/~graebe}}

\date{October 2021}
\begin{document}

{\setbeamertemplate{footline}{}
\begin{frame}
  \titlepage
\end{frame}}

\begin{frame}{What is a System?}
  \begin{block}{V. Petrov}
    A \emph{system} is a set of elements that are interconnected and interact
    with each other, forming a unified whole that possesses properties that
    are not already contained in the constituent elements considered
    individually.
  \end{block}

  \begin{block}{TRIZ Ontology Project}
    The necessity of the use of the term "system" occurs when it is required to
    emphasize that something is large, complex, immediately not wholly
    comprehensible, but at the same time a unified whole.

    Unlike the notions "set" or "aggregate", the concept of a system emphasises
    the ordering, the integrity, the regularity of construction, functioning and
    development. 
  \end{block}
\end{frame}

\begin{frame}{What is a System?}
  \begin{block}{English Wikipedia on System Engineering}
    At its core, systems engineering utilizes systems thinking principles to
    organize this body of knowledge.

    The individual outcome of such efforts, an engineered system, can be defined
    as a combination of components that work in synergy to collectively perform
    a useful function. 
  \end{block}
  
In all these definitions, the \emph{structuredness} and thus
\emph{decomposability} of the system in the analytic dimension is emphasised
on the one hand, and the \emph{interdependence} and thus
\emph{indecomposability} in the execution dimension on the other.

\end{frame}

\begin{frame}{What is a System?}
In the assembled system in addition to the components, the \emph{connecting
  elements} also play an important role.  They mediate the \emph{flow of
  energy, material and information} that is required for the operation of each
component.

\begin{block}{}
  Viable components deliver processual services in \emph{guaranteed} quantity
  and quality during operation, if the \emph{external operational conditions}
  are guaranteed.
\end{block}

These processual services of the components altogether in combination form the
emergent function of the overall system.

\begin{block}{Self-similarity of the concept}
  Components can be considered as systems, where the upper system guarantees
  the required for operation throughput of energy, substance and information.
\end{block}
\end{frame}

\begin{frame}{What is a System?}
  \begin{block}{
    System concept as description of complicated real-world phenomena by
    reduction to the essentials.}
    \begin{itemize}
    \item[(1)] Outer demarcation of the system against an \emph{environment},
      reduction of these relationships to input/output relationships and
      guaranteed throughput.
    \item[(2)] Inner demarcation of the system by combining subareas to
      \emph{components}, whose functioning is reduced to “behavioural control” via
      input/output relations.
    \item[(3)] Reduction of the relations in the system itself to “causally
      essential” relationships.
    \end{itemize}
  \end{block}

\end{frame}

\begin{frame}{What is a System?}
  \begin{block}{
    Such a reductive description (explicitly or implicitly) exploits output from
    prior life:}
    \begin{itemize}
    \item[(1)] An at least vague idea about the (working) input/output services of
      the environment.
    \item[(2)] A clear idea of the inner workings of the components (beyond the
      pure specification).
    \item[(3)] An at least vague idea about causalities in the system itself, that
      precedes the detailed modelling.
    \end{itemize}
  \end{block}

\end{frame}

\begin{frame}{System, Components, Reuse}
  For technical systems additionally the aspect of reuse plays an important
  role.
  
This does not apply, at least on the artifact level, to larger technical
systems – these are unique specimen, even though assembled using standardised
components.

Use of components produced by third parties (components off the shelf) is
widespread in developed engineering.

This ultimately turned the manufacturing of tools and products from an art first
to a craft and later to an industrial process.

\begin{block}{}
  Hence the \emph{production by independent third parties} and thus the
  technical-economic interrelationships of an industrial mode of production
  based on the division of labour are to be considered.
\end{block}

\end{frame}

\begin{frame}{System, Components, Reuse}
  In such a context, the concept of a technical system is fourfold overloaded.

  \begin{block}{A technical systems can be considered}
    \begin{itemize}
    \item [1.] as a real-world unique specimen (e.g. as a product or a service),
    \item [2.] as a description of this real-world unique specimen (e.g. in the
      form of a special product configuration)
    \end{itemize}
  \end{block}

  \begin{block}{and for components produced in larger quantities also}
    \begin{itemize}
    \item [3.] as description of the design of the system template (product
      design) and
    \item [4.] as description and operation of the delivery and operating
      structures of the real-world unique specimens of this system produced
      according to this template (as production, quality assurance, delivery,
      operational and maintenance plans).
    \end{itemize}
  \end{block}
\end{frame}

\begin{frame}{Socio-Technical Systems}

Systems and social practices of cooperatively acting people.

Systems and purposes.  Useful functions.

Problematisation of purposefulness (Zweckmäßigkeit) in "natural", in
particular socio-ecological systems.

\begin{block}{Purposefulness ...}
  ... transforms the totality of technical systems into an interconnected
  \emph{world of techical systems} full of preconditions and conditionalities,
  which opens up a fourth dimension of the concept of system, to secure stable
  operating conditions of the systems themselves.
\end{block}

Socio-technical systems are, in addition to technical restrictions, charged
with the contradictory expectations and interests of concrete people and
groups of people.
\end{frame}

\begin{frame}{Socio-Technical Systems}

  Ian Sommerville \emph{Software Engineering} elaborates a number of
  challenges in this regard.

  \begin{block} {Technical computer-based systems}
    ... are systems that contain hardware and software components, but not
    procedures and processes. ... Individuals and organizations use technical
    systems for specific purposes, but knowledge of that purpose is not part
    of the system.
  \end{block}

\end{frame}

\begin{frame}{Socio-Technical Systems}

  \begin{block}{Socio-technical systems}
    ... contain one or more technical systems, but beyond that -- and this is
    crucial -- the knowledge of how the system should be used to achieve a
    broader purpose.\medskip

    This means that these systems have \emph{defined work processes},
    \emph{human operators} as integral part of the system, are \emph{governed
      by organizational policies} and are \emph{affected by external
      constraints} such as national laws and regulations.
  \end{block}

\end{frame}

\begin{frame}{Essential Characteristics of Socio-Technical Systems}
\begin{itemize}
\item They have special properties that affect the system as a whole, and are
  not related to individual parts of the system. These special properties
  depend on the system components and the relationships between them. Because
  of this complexity, the system-specific properties can only be evaluated
  when the system is composed.\medskip
\item They are often not deterministic. The behaviour of the system depends on
  the human operators and on other people who do not always react in the same
  way. Also, the operation of the system can change the system itself.
\end{itemize}
\end{frame}

\begin{frame}{Essential Characteristics of Socio-Technical Systems}
\begin{itemize}
\item The extent to which the system supports organizational goals depends not
  only on the system itself. It also depends on the \emph{stability of the
    goals}, the relationships and \emph{conflicts between organizational
    goals}, and how people in the organization \emph{interpret those goals}.
\end{itemize}

\begin{block}{}
  In this context, there is a clear shift on the scale of controllability from
  direct control by external human operators to indirect control and movement
  according to intrinsic laws, which is even more prevalent in
  \textbf{socio-economic systems} with a large number of stakeholders or even
  \textbf{socio-ecological systems}.
\end{block}
\end{frame}

\begin{frame}{Shchedrovitsky on Systems Analysis}
  
  The system concept serves to delimit a part of the complex, all-connected
  world (\emph{reality}) in order to make this part accessible for
  description.

However, this human activity, which Georgi Shchedrovitsky (a Russian
Philosopher and the head of the \emph{Methodological School of Management})
refers to as \emph{mental activity} (Denktätigkeit), is itself part of that
reality and thus also of practical relevance. Real-world processes are thereby
charged with description forms.

Thus in system theory these two dimensions -- description and operation --
must be distinguished. Charging a system with a description form is what
Engels' calls, in reference to Kant's \emph{thing in itself}, the
transformation of the \emph{thing in itself} into a \emph{thing for us}.

This concept is developed by Shchedrovitsky in greater detail.
\end{frame}

\begin{frame}{Theory of Dynamical Systems}

The processual dimension of systems can be investigated with the mathematical
tools of the Theory of Dynamical Systems if the processes can be modelled as
equations of motion in phase space.

Phase space and equations of motion. Notion of trajectory.

Examples:\small
\begin{itemize}
\item Pendulum:\\ \url{https://en.wikipedia.org/wiki/Pendulum_(mechanics)}
\item Two body problem:\\ \url{https://en.wikipedia.org/wiki/Two-body_problem}
\end{itemize}
\end{frame}

\begin{frame}{How Chaotic can Trajectories be?} 

Examples: 
\begin{itemize}
\item Double Pendulum, \url{https://en.wikipedia.org/wiki/Double_pendulum}
\item Magnetic pendulum with three attracting magnets,
\item 3-body model: \url{https://en.wikipedia.org/wiki/Three-body_problem}
\end{itemize}

Not everything that looks like chaos has to be chaotic:\\
\url{https://i.redd.it/zr7tet9mdfl01.gif}

\end{frame}

\begin{frame}{Attractors}

Examples: 
\begin{itemize}
\item Pendulum,
\item pendulum with three attracting magnets,
\item pendulum with one repelling magnet.
\end{itemize}

Limit cycles: \url{https://en.wikipedia.org/wiki/Limit_cycle}

When the body is on the limit cycle, it remains there, i.e. the limit cycle is
a \emph{stable solution} of the equations of motion of the system, called
\textbf{steady-state equilibrium}.
    
\end{frame}

\begin{frame}{Attractors}

In many cases the real movement $f(t)$ in time is \emph{attracted} by that
limit cycle, i.e. $f(t)$ can be decomposed into $f(t)=l(t)+r(t)$ with $l(t)$
the projection on the limit cycle and $r(t)\to 0$ a (small) orthogonal
deviation.  In this way, it is often possible to simplify complicated models.

An \textbf{attractor} is a specific steady-state equilibrium with just this
attracting property.

\url{https://en.wikipedia.org/wiki/Attractor}

\end{frame}

\begin{frame}{How Complicated can an Attractor be?}

\begin{itemize}
\item \url{https://en.wikipedia.org/wiki/Attractor}
\item \url{https://en.wikipedia.org/wiki/Lorenz_system}
\item \url{https://de.wikipedia.org/wiki/Lorenz-Attraktor}
\item "Almost all initial points will tend to an invariant set – the Lorenz
  attractor – a strange attractor, a fractal, and a self-excited attractor"
  (Wikipedia)
\end{itemize}

\end{frame}

\begin{frame}{Dissipative Systems}

  Closed and Open Systems

  Relation between a (stable) external throughput of energy, matter and
  information and the inner structure formation in systems.

  \begin{itemize}
  \item \url{https://en.wikipedia.org/wiki/Rayleigh-Benard_convection}
  \item \url{https://en.wikipedia.org/wiki/Belousov-Zhabotinsky_reaction}
  \item Dissipative systems
    \url{https://en.wikipedia.org/wiki/Dissipative_system} 
  \item Life on Earth as a dissipative system.
  \end{itemize}
\end{frame}
\end{document}
