\documentclass{beamer}
\usepackage{lsfolien}
\usepackage[english]{babel}
\usepackage[utf8]{inputenc}

\myfootline{System Modelling and Semantic Web -- Winter Term 2021}{Hans-Gert
  Gräbe}

\def\lt{\texttt{<}}
\def\gt{\texttt{>}}
\def\deg{$^\circ$C}

\newcommand{\ueberschrift}[1]{\begin{center}\bf #1\end{center}}

\title{Modelling Sustainable Systems\\ and Semantic Web\\[6pt]
  Modelling of Conceptual Worlds
  \vskip1em}

\subtitle{Lecture in the Module 10-202-2309\\ for Master Computer Science}

\author{Prof. Dr. Hans-Gert Gräbe\\
\url{http://www.informatik.uni-leipzig.de/~graebe}}

\date{November 2021}
\begin{document}

{\setbeamertemplate{footline}{}
\begin{frame}
  \titlepage
\end{frame}}

\section{Conceptual Worlds}
\begin{frame}{RDF -- Language Forms and Practices}

\ueberschrift{Procedural knowledge $\to$ Practical procedures} 
\begin{itemize}
\item Correspondence between the coherence of the language form and the
  coherence of practices.
\item Establishing coherent practices as procedures allows to substantiate
  predicates.  “At the beginning of the school lesson the sandwiches have to
  be packes away”.  Normative sentences are possible only after such a
  transposition of the predicate to the subject position.
\item Parallels to the concert example in the first lecture.
\item \emph{Processual knowledge} is the description form, \emph{practical
  procedures} are the embodiment.
  \end{itemize}
\end{frame}

\begin{frame}{The Linked Open Data Cloud}
\textbf{Back to the basic idea}.
\begin{itemize}
\item This creates a globally networked \textbf{decentralized open database},
  the Linked Open Data Cloud, in which all \textbf{public} information is
  freely and machine-readable available.
  \begin{itemize}
  \item The collection of data is reversed. Data is nothing private but part
    of a world built around a core stock of \textbf{publicly available
      information} in a \textbf{public domain} as an essential cultural 
    constituent.
  \item See \url{http://lod-cloud.net/}
  \item Growing the LOD Cloud: \url{http://lod-cloud.net/versions/}
  \end{itemize}
\item The context of Industry 4.0 and all major data projects including Google
  is inconceivable without these efforts.
\item Basis for internal and inter-company information systems such as ERP and
  CRM.
\end{itemize}
\end{frame}

\begin{frame}{Namespaces and Conceptual Worlds}
Communication is made possible through the introduction of \textbf{namespaces}
supported as a URI prefix.

Namespaces allow to generate URIs without overlapping.
\begin{itemize}
\item This can be used to generate descriptions that contain the fictions of
  MY world, MY concepts, I-core, worlds and reality, construction of reality,
  without having to transcend MY context mentally.
\item But we want more: cooperation with specific others.
\item Semantics = pragmatically contextualized \textbf{formation of models} as
  a basis for \emph{common practical procedures}.
\item Language is required to \emph{speak about the models themselves},
  and thus ways of formalizing semantics are required, too.
\end{itemize}
\end{frame}

\begin{frame}{Conceptualizations and Conceptual Worlds}
But: \textbf{The Tower of Babel phenomenon}
\begin{itemize}
\item What does it mean that every communicative context along with his own
  practical procedures develops also his own models and speaks his own
  language?
\item Which concepts can support translation services?
\item \textbf{Ontologies} (or vocabularies): WE agree on the usage of common
  namespaces (foaf, skos, org, sioc etc.) for \emph{special} purposes and thus
  on common \emph{partial models} of the world.
  \begin{itemize}
  \item Phenomenon of coherence between private and cooperative language
    practices.
  \end{itemize}
\item How does that work exactly?
  \begin{itemize}
  \item \textbf{Content:} Pragmatically contextualized formation of
    \emph{models} as basis for common \emph{practical procedures}.
  \item \textbf{Form:} \emph{Ontologization} as pragmatically contextualized
    semantification of syntax.
  \item Examples on the next slide.
  \end{itemize}
\end{itemize}
\end{frame}

\begin{frame}{Conceptualizations and Conceptual Worlds}
\textbf{Example foaf: Friend of a Friend}
\begin{itemize}
\item foaf: {\lt}\url{http://xmlns.com/foaf/0.1/}{\gt}
\item Redirect to \url{http://xmlns.com/foaf/spec/}
\item We study the model developed there and the descriptions of semantics and
  syntax.
\end{itemize}

\textbf{Example skos: Simple Knowledge Organization System}
\begin{itemize}
\item skos: {\lt}\url{http://www.w3.org/2004/02/skos/core\#}{\gt}
\item Forwards to a tabular overview
  \url{https://www.w3.org/2009/08/skos-reference/skos.html}.
\item At the very bottom of the page three references with more detailed
  explanations about semantics.
\end{itemize}
\end{frame}

\begin{frame}{Conceptualizations and Conceptual Worlds}
\textbf{Example org: The Organization Ontology}
\begin{itemize}
\item org: {\lt}\url{http://www.w3.org/ns/org\#}{\gt}
\item Forwards to a turtle file. Download and inspect it.
\item rdfs:seeAlso {\lt}\url{https://www.w3.org/TR/vocab-org/}{\gt}
\end{itemize}

A socially extremely difficult process, but that is \textbf{the core of
  semantic technologies}: The \emph{institutionalization} of machine-readable
common conceptual worlds as \emph{social process}.

\end{frame}

\begin{frame}{Conceptualizations and Conceptual Worlds}

Associated with this are \emph{formation of models}, conditionalities
(contextuality of different realities) and the process of transcending
contexts if ontologies are not applied as originally intended.
\begin{itemize}
\item Talk to each other -- agree on ontologies.
\item Further development of ontologies.
\item Large databases of ontologies: \url{http://prefix.cc} or
  \url{http://lov.okfn.org} (Linked Open Vocabularies)
\item Creativity in a cooperative context. Requirement of formalization to
  exchange information as data.  Again the concert example.
\end{itemize}
\end{frame}

\begin{frame}{Conceptualizations and Conceptual Worlds}
Example: DBPedia -- extract structured information from Wikipedia
\begin{itemize}
\item DBpedia is a crowd-sourced community effort to extract structured
  information from Wikipedia and make this information available on the
  Web. ... We hope that this work will make it easier for the huge amount of
  information in Wikipedia to be used in some new interesting ways. ...
\item Example: \url{http://dbpedia.org/page/Leipzig}
\end{itemize}
\end{frame}

\begin{frame}{Conceptualizations and Conceptual Worlds}
Example: Linked data service of the German National Library
\begin{itemize}
\item The German National Library offers a linked data service for long-term
  use of the entire national bibliographic data including all norm data by the
  Semantic Web Community. Offering this data service it endeavors to make a
  contribution to the worldwide information infrastructure as a prerequisite
  for modern commercial and non-commercial web services.
\item \url{http://www.dnb.de/lds}
\end{itemize}
\end{frame}

\begin{frame}{Other Approaches}
\ueberschrift{Schema.org}
\begin{itemize}
\item Other approach: \url{http://schema.org} -- Google's ontologization of
  the world and incorporation into websites instead of building a distributed
  database as the Linked Open Data Cloud.
\item Schema.org and Microdata: \url{https://schema.org/docs/gs.html}.
  \begin{itemize}
  \item itemscope, itemtype and itemprop and the link to RDF.
  \end{itemize}
\item Labeling websites with this markup increases their visibility on Google.
\end{itemize}
\end{frame}

\begin{frame}{Other Approaches}
\ueberschrift{Google's Knowledge Graph}

\textbf{Google's Knowledge Vault:} Extracted facts by supervised learning from
the examined websites as Google's knowledge base.
\begin{itemize}
\item Per 2014 it contained over 1.6 billion facts. Facts were assessed with a
  probabilistic confidence value.
\end{itemize}
\textbf{Google Knowledge Graph:} Consolidation and enrichment with structured
facts from Freebase (founded in 2007, 2010 bought by Google), Wikipedia and
Wikidata.
\begin{itemize}
\item Contained over 70 billion facts per 2016.
\item At the end of 2015, the Google Knowledge Graph API was published. Web
  developer can use it to access the stock of data.
\end{itemize}\vspace*{2em}
\end{frame}

\begin{frame}{Other Approaches}
But that is only a part of the \textbf{Giant Global Graph} (Tim Berners-Lee,
2007).
\ueberschrift{Wolfram Alpha}

Also a \emph{search engine} that is based on facts from own collection of
data.  Uses Mathematica as additional \emph{compute engine} to create more
complex presentations and visualizations.

The goal is to network mathematical knowledge and general knowledge.
\begin{itemize}
\item \url{https://www.wolframalpha.com}
\item Example "Leipzig".
\end{itemize}
\end{frame}
\end{document}
