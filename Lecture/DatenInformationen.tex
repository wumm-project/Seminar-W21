\documentclass{beamer}
\usepackage{lsfolien}
\usepackage[english]{babel}
\usepackage[utf8]{inputenc}

\myfootline{System Modelling and Semantic Web -- Winter term 2021}{Hans-Gert
  Gräbe}

\newcommand{\ueberschrift}[1]{\begin{center}\bf #1\end{center}}

\title{Modelling Sustainable Systems\\ and Semantic Web\\[6pt] Data and
  Information \vskip1em}

\subtitle{Lecture in the Module 10-202-2309\\ for Master Computer Science}

\author{Prof. Dr. Hans-Gert Gräbe\\
\url{http://www.informatik.uni-leipzig.de/~graebe}}

\date{December 2021}
\begin{document}

{\setbeamertemplate{footline}{}
\begin{frame}
  \titlepage
\end{frame}}

\section{The Internet as a World of Shortcuts}
\begin{frame}{The Internet as a World of Shortcuts}
  
\ueberschrift{Data and information -- a first approximation}

Bit streams and \textbf{data packets}.
\begin{itemize}
\item There are no bit streams on the "Internet", but rather data packets that
  are sent and received at the devices.  Data packets are generated and
  transformed back again from bit streams at the 4 lower levels of the OSI
  stack.
\item Shortcut of speaking about the universally networked end devices and
  reality of the net failures.
\end{itemize}
The mouse phenomenon
\begin{itemize}
\item Tools and their use. The spoon.
\item Internalisation and processual skills. 
\item Shortcuts of speaking in everyday life. \textbf{Discussion}.
\end{itemize}\vspace*{2em}
\end{frame}

\begin{frame}{The Internet as a World of Shortcuts}

\begin{block}{The Notion of Shortcut}
  \textbf{Shortcut} as socially supported, guaranteed and sustained
  \emph{consensus} on an \emph{abbreviated way of speaking} about a
  \emph{social normality}.
\end{block}
\begin{itemize}
\item Shortcuts of speaking are a specific way of dealing with a increasing
  complexity of the world.
\item Shortcuts in this sense are not a new phenomenon.
\item Shortcuts are close related to systemic thinking.
\item Shortcuts and Myths
  \begin{itemize}
  \item A \emph{myth} in its original meaning is a story. The religious myth
    links human existence with the world of gods or ghosts. Myths demand to be
    valid for the truth they claim. ... The ensemble of all myths of a nation,
    a culture, a religion is called \emph{mythology}.  (Wikipedia)
  \end{itemize}
\end{itemize}
\end{frame}

\begin{frame}{The Internet as a World of Shortcuts}
\ueberschrift{Complexity and Clock Frequencies in a Society}
\begin{itemize}
\item A clock (or timing) is used to impress a periodicity to a sequence or to
  synchronise processes. The system clock in a computer determines the working
  speed of many components. (Wikipedia)
\item Timing is also essential for the coordination and synchronization of
  social activities.
\item Development of complexity and clock rates of computer chips see
  \url{https://arxiv.org/pdf/1803.00254.pdf}
\item Moore's Law (1965) states that complexity of integrated circuits with
  minimal component costs doubles on a regular basis. Depending on the source,
  the period is 12 to 24 months.
\item But the „human clock rate“ does not change ...
\end{itemize}\vspace*{2em}
\end{frame}

\begin{frame}{Theoretical Expansion of a Shortcut}

  \textbf{Shortcut:} universal end-to-end connectedness in the internet.

  Theoretical reflection as a scale-free network\small
\begin{itemize}
\item $v(k)=c\cdot k^{-a}$ -- proportion of nodes with $k$ neighbors ($v$ as
  valence). \item Example with $a=3$: $v(1)=0.832$, $v(2)=0.104$,
  $v(3)=0.031$, $v(4)=0.013$, $v(5)=0.007$, $v(6)=0.004$, ...
\item Compared to a random network (another model!) the proportion of
  nodes with many connections (hubs) decreases slower.
\item Scale-free networks are robust against the failure of a larger number of
  randomly selected nodes, but not against failure of a small number of hubs.
\item Robustness: Each node is embedded in a local socio-technical
  infrastructure, which takes care of its operation, maintains the "social
  normality" and thus reproduces the socio-technical conditions of this
  shortcut of speaking.
\end{itemize}\bigskip
\end{frame}

\section{Data and Information}
\begin{frame}{Data and Information}

  \begin{center}
    \includegraphics[width=\textwidth]{images/NuqlkG.png}
  \end{center}
\end{frame}

\begin{frame}{Data and Information}

  \ueberschrift{Syntax, Semantics, Pragmatics}
  
  \begin{block}{Data and Information. A first definition}
    \begin{quote}
      Information = interpreted data\\
      Data = formalized information
    \end{quote}
  \end{block}

Both (formalization and interpretation) are only "valid" in a special natural,
technical or social \emph{embedding} -- a \emph{context} (or pragmatics) --
and thus \textbf{assume} the existence of a "working shortcut of speaking".

Compare this also with the concert example in the first lecture.
\vfill
\end{frame}
\begin{frame}{Data and Information}

  \ueberschrift{Syntax, Semantics, Pragmatics in the OSI layer model}

We consider such a \emph{pragmatically} contextualized interplay of
(formalized) \emph{syntax} and (formalized) \emph{semantics} on different
levels at the example of the OSI stack.

\begin{itemize}
\item Each layer is based on a shortcut of speaking (i.e., social normality)
  and its language representation given as formalized syntax.
\item This formalized syntax was practically produced on the previous layer
  as \emph{components} within the newly emerging system.
\item On this basis a further pragmatics is realised through language
  constructions as special way of speaking (semantics) about \emph{relations}
  in the newly emerging system.
\item This special way of speaking in turn is formalized for use on the next
  layer (the next system level).
\end{itemize}\bigskip
\end{frame}

\begin{frame}{Data and Information}
  \begin{center}
    \includegraphics[width=\textwidth]{images/Rqw330.jpg}
  \end{center}
Source: Wikipedia, \url{http://prima-it.de/images/osi7layermodell.jpg}
\end{frame}
\begin{frame}{Data and Information}

  \ueberschrift{Syntax, Semantics, Pragmatics in the OSI layer model}

\emph{Explanation of this idea:}

Layer 1: 
\begin{itemize}
\item Syntax = modulated waves,
\item Semantics = bit sequences (first shortcut of speaking), 
\item Pragmatics = diversity of transmission media
\end{itemize}
Layer 2: 
\begin{itemize}
\item Syntax = bit sequences, 
\item Semantics = frames (second shortcut of speaking),
\item Pragmatics = control of the transmission speed of the bit sequences,
  addition of checksums for error detection
\end{itemize}\vspace*{2em}
\end{frame}
\begin{frame}{Data and Information}

  \ueberschrift{Syntax, Semantics, Pragmatics in the OSI layer model}

Layer 3: 
\begin{itemize}
\item Syntax = frames, 
\item Semantics = data packets (third shortcut of speaking),
\item Pragmatics = routing and organization of forwarding of packets across
  multiple nodes
\end{itemize}
Etc.\vfill
\end{frame}
\end{document}
