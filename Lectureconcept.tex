\documentclass[11pt,a4paper]{article}
\usepackage{ls}
\usepackage[english]{babel}

\setcounter{secnumdepth}{-2}

\title{Concept of the Lecture \\[1em] \emph{Modelling Sustainable Systems and
    Semantic Web} \\[1em] Winter Term 2021}

\author{Hans-Gert Gr\"abe}

\date{October 3, 2021}

\begin{document}
\maketitle

\subsection{General}

The lecture will take place synchronously online (Thursdays 11-13 am.) and is
based on the Flipped Classroom concept. The lecture consists of three parts.

In the first part we explore the concept of a \emph{technical system} and
introduce the main concepts of TRIZ as an important systematic innovation
methodology.  In contrast to other creativity and innovation methodologies,
TRIZ focuses on the systematisation of engineering experiences.

In the second part, we study more closely aspects of the creation of
conceptual networks for data models on the basis of the \emph{Resource
  Description Frameworks} (RDF), the \emph{Linked Open Data Cloud}, the
emerging \emph{Giant Global Graph} and the importance of these developments
for the organisation of contexts of cooperative action and hence management of
sustainable systems.

Finally, in the third part, we explore the role of data and information and
the generation of new language tools for the development of technical systems
in the context of a civil society and, in particular, the importance of
concept formation processes in cooperative action.

In addition to the general bibliography, each lecture will be accompanied by
literature for preparation which should be \textbf{studied before the
  lecture}, in order to be able to follow the explanations. In the lecture the
topics are presented only cursorily, but there is room to ask questions about
the literature and to discuss individual aspects.

Most of the material is easily found on the internet. Nevertheless we will not
dispense on classical printed literature and your ability to access it.

To support our opening for an international audience, we switch step by step
to English as operational language. The lecture will use a mixed form, with
English slides and German presentation.

The progress of the lecture will be reported regularly in the github
repository \url{https://github.com/wumm-project/Seminar-W21}. There you also
find the schedule and the slides of the individual lectures.

\subsection{Digital Privacy}

We follow not only a theoretical but also a practical Open Culture approach
and make course materials publicly available.  This also applies to the
(annotated) chat recordings of the lecture, in which your names are mentioned.
We assume your consent to this procedure, if you do not explicitly object.
The discussions themselves will \textbf{not} be recorded.

\subsection{Literature}

\begin{itemize}
\item Robert Adunka (2020). TRIZ Anwendungsbeispiele (in German)\\
  \url{https://www.triz-consulting.de/ueber-triz/triz-anwendungsbeispiele-2/} 
\item Iouri Belski (2020). Tools of TRIZ. A web repository of TRIZ materials
  on 12 simple TRIZ heuristics.
  \url{https://emedia.rmit.edu.au/triz/content/tools-triz}
\item Peter L. Berger, Thomas Luckmann (1966). The Social Construction of
  Reality: A Treatise in the Sociology of Knowledge. Anchor Books. ISBN
  978-0-385-05898-8. \\ (In German: Die gesellschaftliche Konstruktion der
  Wirklichkeit: eine Theorie der Wissenssoziologie. Fischer Taschenbuch Verlag
  1994. ISBN 978-3-596-26623-4).
\item Raphael Capurro, Peter Fleissner, Wolfgang Hofkirchner (2000). Is a
  unified theory of information feasible? A Trialogue.
  \url{http://www.capurro.de/trialog.htm}
\item Gaetano Cascini (2012). TRIZ-based Anticipatory Design of Future
  Products and Processes. Journal of Integrated Design and Process Science 16
  (3), 29--63.\\  \url{http://dx.doi.org/10.3233/jid-2012-0005}
\item Frank W. Geels, Johan Schot (2007). Typology of Sociotechnical Transition
  Pathways. In: Research Policy 36 (2007), 399–417.\\
  \url{https://doi.org/10.1016/j.respol.2007.01.003} 
\item Hans-Gert Gräbe (2020). Man and its technical systems (in German).
  LIFIS Online.\\ \url{http://dx.doi.org/10.14625/graebe_20200519}
\item Hans-Gert Gräbe (2020). TRIZ and transformations of socio-technical and
  socio-ecologi\-cal systems (in German). LIFIS Online.\\
  \url{http://dx.doi.org/10.14625/graebe_20200627}
\item C.S. Holling (2000). Understanding the Complexity of Economic,
  Ecological, and Social Systems. In: Ecosystems (2001) 4, 390–405.
  \url{https://www.esf.edu/cue/documents/Holling_Complexity-EconEcol-SocialSys_2001.pdf}
\item Helmut Klemm (2003). Ein großes Elend. Informatik-Spektrum 26,
  S. 267–273. \\ \url{http://dx.doi.org/10.1007/s00287-003-0316-2}
\item Karl Koltze, Valeri Souchkov (2017). Systematische Innovationsmethoden
  (in German).  Hanser Verlag, München. ISBN 9783446451278
\item Andrei Kuryan, Dmitri Kucharavy (2018). The OTSM-TRIZ Heritage of
  Nikolai N. Khomenko. A General Theory of Powerful Thinking. Slides of a talk
  given at TDS 2018 in St. Petersburg.  \\
  \url{http://www.informatik.uni-leipzig.de/~graebe/Material/OTSM-Folien.pdf}
\item Nikolai Khomenko, John Cooke (2007). Inventive problem solving using the
  OTSM-TRIZ “TONGS” model.\\
  \url{http://www.informatik.uni-leipzig.de/~graebe/Material/tongs-en.pdf}.
\item Alex Lyubomirskiy, Simon Litvin, Sergei Ikovenko et al. (2018). Trends
  of Engineering System Evolution (TESE).  TRIZ Consulting Group. ISBN
  9783000598463.
\item Michael Schetsche (2006). Die digitale Wissensrevolution –
  Netzwerkmedien, kultureller Wandel und die neue soziale Wirklichkeit (in
  German). In: zeitenblicke 5 (2006), Nr.~3.\\
  \url{http://www.zeitenblicke.de/2006/3/Schetsche} 
\item Ian Sommerville (2015). Software Engineering. 10th edition.  (in German:
  Pearson Studium, 8. Auf\-lage, 2007).\\
  \url{http://iansommerville.com/software-engineering-book/}
\item Valeri Souchkov (2010).  TRIZ and Systematic Business Model Innovation.
  In: Proceedings TRIZ Future Conference 2010, Bergamo, Italy.  Available at
  ResearchGate.
\item Valeri Souchkov (2014).  Breakthrough Thinking with TRIZ for Business
  and Management: An Overview.
  \url{http://www.xtriz.com/TRIZforBusinessAndManagement.pdf}
\item Felix Stalder (2016). Kultur der Digitalität (in German). Suhrkamp.
\item Clemens Szyperski (2002). Component Software. Pearson Education.
  2. Auflage.  ISBN 0201745720.
\item Rainer Thiel (2000). Die Allmählichkeit der Revolution. Blick in sieben
  Wissenschaften (in German).  LIT-Verlag, Münster. ISBN 9783825849457.
\item Dietmar Zobel (2007). Kreatives Arbeiten (in German). Expert Verlag,
  Renningen.\\ ISBN 9783816927136.
\item Dietmar Zobel (2020). TRIZ für alle (in German). Expert Verlag,
  Renningen. ISBN 9783816985105.
\end{itemize}

\end{document}
