\documentclass[11pt,a4paper]{article}
\usepackage{ls}
\usepackage[english]{babel}

\setcounter{secnumdepth}{-2}

\title{Concept for the Research Seminar\\ "Sustainability, Environment,
  Management"}

\author{Hans-Gert Gr\"abe, Ken Pierre Kleemann, Sabine Lautenschläger (Uni
  Leipzig),\\ Ralf Laue, Kristin Kutzner (WH Zwickau) }

\date{October 9, 2021}

\begin{document}
\maketitle

\section{Aim and methodology of the seminar}

The concept of a \emph{system} plays a prominent role in computer science when
it comes to database systems, software systems, hardware systems, accounting
systems, access systems, etc.  In general, computer science is regarded by a
majority as the "science of the \emph{systematic} representation, storage,
processing and transmission of information, especially their automatic
processing using digital computers" (German Wikipedia).  Also certain relevant
professions such as the \emph{system architect} are in high esteem by IT
users.

However, the significance of the concept of system extends far beyond the
field of computer science -- it is fundamental for all engineering sciences
and as \emph{Systems Engineering} with the ISO/IEC/IEEE-15288 standard
"Systems and Software Engineering", it is also the subject of international
standardisation processes.  Even more, the concept of systems also plays an
important role in the description of complex natural and cultural processes --
for instance in the concept of an \emph{ecosystem}.

While classical TRIZ focuses strongly on instrumentally feasible engineering
solutions, Systems Engineering "is an interdisciplinary field of engineering
and engineering management that focuses on how to design, integrate, and
manage complex systems over their life cycles. At its core, systems
engineering utilizes systems thinking principles to organize this body of
knowledge. The individual outcome of such efforts, an engineered system, can
be defined as a combination of components that work in synergy to collectively
perform a useful function." (English Wikipedia). 

Earlier in this seminar, we had already studied more intensively different
system concepts and, in particular, examined their application in complex
socio-ecological, socio-economic and socio-technical contexts, see
\cite{Graebe2020}. We observed that the central concepts of \emph{transition
  management} and \emph{activity management} addressed two different
perspectives on structural change processes. In the transition management
approach, the structural-transitional challenges are in the foreground, the
activity management approach studies the implementation of structural changes
via the actions and co-actions of actors and stakeholders.

In both approaches, however, the focus was on a holistic-structural and
analytical view of a \emph{decision preparation} rather than on practical
procedural management approaches of \emph{decision-making} and decision
implementation in complex and contradictory real-world situations.

The WUMM project\footnote{WUMM stands in German for \emph{Widersprüche und
    Managementmethoden} (Contradictions and Management Methods).} aims at a
better understanding of such management processes. Our starting point is TRIZ
as a systematic innovation methodology derived from engineering experience in
contradictory requirement situations. With the field of "Business TRIZ", which
has been unfolding for about 20 years, a transfer of experience is being
actively promoted, embedded in older management cultures and theories.  A
better understanding of such approaches to management issues and their
connection to systemic concepts and approaches was in the focus of our seminar
last semester.

In recent years, co-operative action \cite{Goodwin2018, Krug2019} by
differently specialised experts has become increasingly important.  In such
interdisciplinary work contexts, the development of \emph{common conceptual
  systems} of sufficient performance proves to be a difficult problem that can
be supported by digital semantic technologies.  Parallel to these challenges
\emph{agile approaches} play a major role, not only in the field of
management, but also increasingly in the solution of socio-technical and
engineering problems concerning ongoing co-operative actions in
multi-stakeholder contexts -- for example with the concept of \emph{technical
  ecosystems}.

\textbf{In the seminar}, we want to learn more about such modern management
appoaches in which \emph{common conceptualisations} and
\emph{consensus-oriented decision-making processes} are central and of crucial
importance for the success and ways of formation and consolidation of new
systemic structures.  We are particularly interested in the connection between
the dialectical resolution of contradictory requirement situations in the
sense of TRIZ methodology and the emergence of common conceptual and
notational worlds as a result of the application of suitable semantic web
technologies.  A special emphasis will be put on the work of the
\emph{Methodological School of Management} and the Moscow Methodological
Circle around G.P. Shchedrovitsky \cite{Khristenko2014, Shchedrovitsky1981}.

The seminar is a \textbf{research seminar} in which we jointly explore
different aspects of co-operative action in different management concepts.
With this seminar, we are approaching a topic that is new to us, which offers
the opportunity to participate in a joint academic explorative process on a
basis of equals. This bears opportunities, but also challenges.  The students
are expected to actively participate in the seminar through seminar
discussions, presentations and last but not least by reading the relevant
materials.  For the successful completion of the seminar, a topic has to be
presented as discussion leader and a handout of 2--3 pages on the topic has to
be submitted in advance.

The seminar is accompanied by a \textbf{lecture} \emph{Modelling Sustainable
  Systems and Semantic Web} (Thursdays 11-13 a.m.) in which important concepts
of our interdisciplinary course programme such as
\begin{itemize}[noitemsep]
\item technology as a whole of socially available procedural knowledge,
  institutionalised procedures and private procedural skills, 
\item sustainability requirements in systemic concepts,
\item digital change and concepts of semantic web technologies,
\item concept and knowledge formation processes,
\item cooperative action, network economies and open culture
\end{itemize}
are developed in more detail. The lecture and the seminar are not directly
related to each other, but conceptual frameworks developed in the lecture will
be heavily present in the seminar. There is a slide stack \cite{Graebe2021}
available from the lecture in the previous semester.

All materials and seminar reports that can be made publicly available, will be
published in the github repository
\url{https://github.com/wumm-project/Seminar-W21}.

\section{Seminar Organisation}

The seminar will be held weekly on Tuesdays 9-11 a.m. (Leipzig time)
synchronously online.  Prior to each appointment participants have to study
the assigned reading to be in a position to discuss the problems in the
seminar.  The seminar is moderated by a \emph{discussion leader}, who prepares
a short workout of 2--3 pages and makes it available to the participants in
advance \emph{before the seminar} (by Sunday evening).

Students of Leipzig University find more about the seminar in the Saxonian
e-learning platform
OPAL\footnote{\url{https://bildungsportal.sachsen.de/opal/} -- Course
  W21.BIS.SIM.}.  The platform will be used for organisational purposes only.
The \textbf{primary source for the seminar plan} is the (actual version of
the) file \texttt{Seminarplan.md} in the github repository \emph{Seminar-W21}.

\section{Examination. Topics for Seminar Work}

In order to be admitted to the examination, the seminar must be successfully
completed, one of the seminars has to be moderated as discussion leader and
for this seminar a short workout has to be prepared and made available to the
participants.

Students who are enrolled in the 10-LP module "Semantic Web" must also
successfully complete the TRIZ lab and then take an oral examination (30
minutes) in February 2022 about the acquired knowledge of concepts of
systematic innovation methodologies and Semantic Web.

Students who are enrolled in the 5-LP seminar module "Applied Computer
Science" have additionally to prepare a Seminar Work (about 20 pages) as
examination.  The work has to be completed until the end of the semester on
March 31, 2022.

\section{Privacy}

We follow an Open Culture approach not only theoretically, but also
practically and make course materials publicly available.  This also applies
to the course materials you have to produce (presentations, seminar papers) as
well as to (annotated) chat sessions of the seminar discussions, in which your
names are also mentioned.  We assume your consent to this procedure if you do
not explicitly object.  The seminar discussions themselves are \textbf{not}
recorded.

To simplify the further use of the materials and texts, the papers are asked
to be compiled in English using {\LaTeX}.  Also the {\LaTeX} source should be
provided under the terms of the
CC-BY\footnote{\url{https://creativecommons.org/licenses/by/4.0/}} license in
order to create a corresponding corpus of texts that can be used to accompany
similar efforts in the OpenDiscovery project. Of course, this cannot be
"decreed". \textbf{Please inform the seminar instructor if you do not wish to
  make your work available for exchange under these conditions}.

\section{Seminar plan}

The seminar starts on October 12, 2021 with a kick-off meeting.  The exact
topics and themes will be published at the beginning of the seminar, when the
number of participants can be estimated more precisely.

We assume that student participants will mainly prepare and present on
different management topics. A non-exhaustive list of possible topics is
compiled in the \emph{Seminar Plan}.

\begin{thebibliography}{xxx}
\bibitem{Goodwin2018} Charles Goodwin (2018). Co-operative Action.  Cambridge
  University Press. ISBN 978-1-108-71477-8.
  
  Available as e-book \url{https://doi.org/10.1017/9781139016735} at UB
  Leipzig using your shibboleth credentials at UL.
\bibitem{Graebe2020} Hans-Gert Gräbe, Ken Pierre Kleemann (2020). Seminar
  Systemtheorie. Universität Leipzig. Wintersemester 2019/20 (in German).
  Rohrbacher Manuskripte, Heft 22. ISBN 9783752620023.
\bibitem{Graebe2021} Hans-Gert Gräbe (2021). Slide Stack to the Lecture.
  Leipzig University, Summer Term 2021.\\
  \url{http://www.informatik.uni-leipzig.de/~graebe/skripte/S21-SW-Slides.pdf}
\bibitem{Khristenko2014} Viktor B. Khristenko, Andrei G. Reus, Alexander
  P. Zinchenko et al. (2014). Methodological School of Management. Bloomsbury
  Publishing.  ISBN 978-1-4729-1029-5.

  Available as e-book at UB Leipzig\\
  \url{https://ebookcentral.proquest.com/lib/leip/detail.action?docID=6159470}
\bibitem{Krug2019} Maximilian Krug (2019). Review: Charles Goodwin (2018).
  Co-Operative Action (in German). Forum Qualitative Sozialforschung, 20(1),
  1-7.  \\ \url{https://doi.org/10.17169/fqs-20.1.3197}.
\bibitem{Shchedrovitsky1981} Georgi P. Shchedrovitsky (1981). Principles and
  General Scheme of the Methodological Organization of System-Structural
  Research and Development.  \\
  \url{https://wumm-project.github.io/Texts/Principles-1981-en.pdf}
\end{thebibliography}

\end{document}

In addition to these themes, interest in the following topics was expressed in
advance:
\begin{itemize}
\item Gr\"abe: Methodological School of Management \cite{Khristenko2014,
  Shchedrovitsky1981}.  A compact presentation of the approaches of the Moscow
  Methodological Circle to questions of a systematic management methodology,
  which had considerable influence on the shaping of the TRIZ approaches.
\item Gr\"abe: Co-operative Action \cite{Goodwin2018, Krug2019}. From Krug's
  abstract: "Charles Goodwin is considered one of the pioneers of social
  interaction research. In his latest book he rearranges his previous
  publications in terms of a concept that could lead to a radical turn in
  anthropology, because his conception of co-operative action covers not only
  the practices of moment-by-moment actions in face-to-face interactions.
  Rather, his approach also encompasses actions with so-called absent
  predecessors, whose previous actions in the form of materiality or bodies of
  knowledge left behind have an impact on the actions of the interactants in
  the here and now. \ldots"
\item Gr\"abe: Introduction to Business TRIZ. \cite{Souchkov2010},
  \cite{Souchkov2014}
\item Gr\"abe: ISO 56\,000 -- \emph{Innovation Management — Fundamentals and
  Vocabulary}
\item Lautenschläger: Management of socio-ecological transformations. 
\end{itemize}

\begin{thebibliography}{xxx}
\bibitem{Souchkov2010} Valeri Souchkov (2010).  TRIZ and Systematic Business
  Model Innovation.  In: Proceedings TRIZ Future Conference 2010, Bergamo,
  Italy.   Available at ResearchGate.
\bibitem{Souchkov2014} Valeri Souchkov (2014).  Breakthrough Thinking with
  TRIZ for Business and Management: An Overview.
  \url{http://www.xtriz.com/TRIZforBusinessAndManagement.pdf}
\bibitem{Yu2010} Eric Yu, Paolo Giorgini, Neil Maiden, John Mylopoulos (2010).
  Social Modeling for Requirements Engineering. MIT Press.  ISBN
  978-0262240550.
  
  Available as e-book at UB Leipzig\\
  \url{https://ebookcentral.proquest.com/lib/leip/detail.action?docID=3339201}
\end{thebibliography}

\end{document}

